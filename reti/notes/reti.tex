\documentclass[12pt, a4paper]{article}
\usepackage[utf8]{inputenc}

\usepackage{graphicx}

\usepackage{geometry}

\usepackage{multicol}

\usepackage{amsmath}
\usepackage{amsfonts}
\usepackage{amssymb}

\geometry{margin=0.6in}


\setlength{\parindent}{0em}
\setlength{\parskip}{1em}

\title{Reti}

\begin{document}

\section{Introduzione}

La rete può essere vista come un insieme di terminali, collegamenti e nodi, che utilizzano varie tipologie di comunicazione.

Alcune di queste metodologie sono i sistemi terminali (Dove si collegano tutti a un mainframe), l'architettura client/server 
e la peer-to-peer (che rimuove o limita l'utilizzo di server)

\subsection{Collegamenti}
Le reti di accesso si generalizzano in 3 categorie: accesso residenziale, accesso aziendale e accesso mobile.

L'accesso residenziale punto-punto utilizza tecnologie come la FTTH per fornire linee internet limitate a una residenziale.
Le componenti tipiche sono il modem, il router, il firewall, il NAT e l'access point. Generamente ritroviamo tutte queste 
componenti unite nell'unico "router" casalingo

In un accesso aziendale le tecnologie aumentano, vediamo l'introduzione della LAN e collegamenti via Ethernet

Infine vediamo le tecnologie wireless, che, attraverso un access point, forniscono linea attraverso l'etere

\subsubsection*{Mezzi trasmissivi}
Il mezzo fisico che connette i dispositivi rientra nella categoria dei mezzi guidati o nelle onde libere. Il mezzo guidato
più comune è il filo di rame a doppino intrecciato (TP). Le sigle dei doppini identificano i vari tipi di schermatura, 
la parte sinistra è la schermatura dell'intero cavo (Unshielded, Foiled, Shielded o maglia metallica) e la parte destra 
indica il singolo doppino (Unshielded and Shielded). Questi casi possono anche essere più o meno incrociati (cross o patch)
da un connettore all'altro. Questo permette di connettere direttamente due terminali. 


Per I mezzi a onde libere abbiamo le microonde terrestri, le WiFi LAN, le Wide area e le satelittari. Questo tipo di 
propagazione è più vulnerabile agli effetti dell'ambiente di propagazione

\subsubsection*{Nucleo della rete}
Il trasferimento dei dati nella rete avviene tramite commutazione di circuito o commutazione di pacchetto. Il primo metodo 
era quello classico della rete telefonica, il che comportava l'assenza di condivisione delle risorse. La rete viene suddivisa 
in porzioni con ripartizione della banda tramite divisione in frequenza o tempo. Nel caso le risorse non vengono utilizzate
allora rimangono inattive

\subsection{Struttura di internet}
Internet è strutturato gerarchicamente. Nel punto più alto troviamo gli ISP di liv 1 che forniscono copertura nazionale se 
non addirittura internazionale. Le comunicazioni fra di loro vengono considerate fra pari (peer).
Successivamente l'ISP di livello 1 vende copertura agli ISP di liv 2, che sono in grado di comunicare con altri ISP di 
liv 2 e un numero limitato di ISP liv 1. Infine abbiamo infine le reti di ISP di liv 3 e reti locali. Queste reti vengono 
definite reti di ultimo salto (last hop)

In queste reti sono disponibili gli IXP (Internet eXchange Point), ovvero edifici dove gli ISP di livello 2 comunicano fra 
di loro direttamente

\subsection*{Ritardi e Perdite}
Se troppi pacchetti arrivano in un router che non riesce a processarli in tempi brevi, allora i pacchetti vengono accodati.
Questi problemi avvengono in 4 casistiche

\subsubsection*{elaborazione del nodo}
Questi problemi avvengono a causa del controllo sugli errori del bit o per la scelta del canale d'uscita

\subsubsection*{Ritardo per accodamento}
I pacchetti si fermano nel router in attesa di trasmissione o per congestione del router

\subsubsection*{Ritardo di trasmissione}
Ritardo all'interno di un dispositivo di rete

\subsubsection*{Ritardo di propagazione}
Ritardo dato dalla trasmissione del mezzo trasmissivo tra due dispositivi di rete


Quindi il ritardo del nodo è dato dalla somma di ritardo di elaborazione (processing delay), ritardo di accodamento 
(queuing delay), ritardo di trasmissione (transmission delay) e ritardo di propagazione (propagation delay)
\begin{center}
    $d_{node}=d_{proc}+d_{queue}+d_{trans}+d_{prop}$
\end{center}

Un'altra formula importante è quella del ritaro d'accomodamento, che è dato dalla lunghezza del pacchetto per il tasso 
medio di arrivo dei pacchetti, il tutto fratto il bitrate
\begin{center}
    $\frac{A\times L}{R}$
\end{center}

\subsection{Livelli di protocollo}
Ogni protocollo viene organizzato su vari livelli in base al suo ruolo, questa strategia si chiama stratificazione e 
permette semplificare l'identificazione di un protocollo e le sue funzioni. Inoltre le modifiche a un determinato 
livello risulta trasparente rispetto agli altri layers.

I layer sono in grado di fornire servizi al layer superiore, mentre utilizzano quelle del livello inferiore. Questi 
servizi vengono forniti attraverso i SAP (Service Access Point). Questi layer sono 5 nella struttura Tcp/ISP

\subsubsection*{Applicazione}
Fornisce alle applicazioni i mezzi per scambiarsi i dati. A questo livello le data unit si chiamano messaggi.
\subsubsection*{Transporto}
Gestisce i problemi di qualità del livello di rete, applica la segmentazione e ricomposizione dei dati, multiplexing. 
Applica anche controlli di flusso, errore e riordino dei pacchetti. Le data unit vengono chiamate segmenti
\subsubsection*{Rete}
Responsabile dell'instradamento dei dati tra un host e l'altro. Offre servizi connection-less o connection-oriented 
(l'indipendenza durante l'instadazione tra un pacchetto e l'altro). Le data unit vengono chiamate pacchetti o datagrammi
\subsubsection*{Collegamento}
Si occupa di multiplexing, effettua controlli e correzzioni di errori e implementa il MAC (Medium Access Controll).
Le data unit vengono chiamate frame
\subsubsection*{Fisico}
trasferimento dei singoli bit sul mezzo di comunicazione (elettrico, elettromagnetico, luminoso,...). Fornisce i servizi per
creare, mantenere e distruggere le connessioni fisiche

Nel modello ISO/OSI si aggiungono anche i livelli presentazione e sessione che si occupano di cifrature e 
sincronizzazione

\end{document}