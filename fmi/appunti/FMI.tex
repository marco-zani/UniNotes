\documentclass[11pt, letterpaper]{article}
\usepackage[utf8]{inputenc}
\usepackage{fixltx2e}
\usepackage{amsmath}
\usepackage{amsfonts}
\usepackage{multicol}
\usepackage{amssymb}
\usepackage{geometry}
\usepackage{tabto}
\geometry{margin=1.2in}
\title{Statistica}

\setlength{\parindent}{0em}
\setlength{\parskip}{1em}

\begin{document}
    
\section{First week}
\textbf{2$^{A}$:} insieme delle parti di A $\Rightarrow\ 2^{\wedge\#A}$ = elenco delle parti di A

\textbf{Relazioni:} dati 2 insiemi X e Y, e un sottoinsieme $\mathcal{R}$(X,Y) è detto relazione tra 
X e Y e scriveremo x$\mathcal{R}$y, x,y$\in\mathcal{R}$

\textbf{Funzione:} siano dati X,Y e dia f una relazione tra X e Y, f $\subset X\times Y$.
diremo che f è una funzione da X in Y se vale:
\begin{center}
    $\forall x \in X: \exists ! y \in Y t.c. (x,y)\in f$
\end{center}

\textbf{Dominio:} insieme delle x che vanno in Y

\textbf{Codomidio:} insieme delle y che hanno corrispondenza in X

\textbf{Legge:} proprietà che definisce una relazione da X a Y

\textbf{Insieme di tutte le funzioni:} $Y^{X}$ corrisponde a tutte le funzioni con leggi diverse ma
con stessi insiemi di partenza ed arrivo

\textbf{Funzione identità:} $id_{X}(X) = X$

\textbf{Composizione di funzioni:} $x\rightarrow^{f} y\rightarrow^{g} z \Rightarrow g(f(x)) = z
\Rightarrow gof(x) = z$ 

\textbf{Iniettiva:} ad ogni f(x) corrisponde un solo y\\
\textbf{Surgettiva:} ad ogni y corrisponde un f(x)\\
\textbf{Bigiettiva:} sia iniettiva che suriettiva

\textbf{Inversa:} se f è biettiva, allora esiste $g=f^{-1}$

\newpage
\section{Second week}

\textbf{Sistemi equipotenti:} X e Y sono equipotenti ($X\sim Y$) se hanno la stessa
cardinalità e la funzione $f:X\rightarrow Y$ è bigiettiva (o invertibile)

\textbf{insiemi cardinali:} sono gli insiemi in formato \{0,1,...,n\} equipotenti all'insieme dato, si 
rapprensentano $|A|$ e definiscono una cardinalità pari a n+1

\textbf{TEOREMA:} X e Y sono equipotenti se e solo se i loro insiemi cardinali sono uguali
\begin{center}
    $|X|=|Y|$
\end{center}

\textbf{Numeri naturali:} sono definiti dagli assiomi di Peano:
\begin{itemize}
    \item 0 è un numero naturale
    \item esiste una funzione successivo $\mathbb{N}\rightarrow\mathbb{N}$
    \item succ(n) $\in\mathbb{N}\backslash\{0\}$, cioé il successivo di ogni naturale è diverso da 0
    \item vale principio d'induzione
\end{itemize}

\textbf{Principio d'induzione:} con $A\subset\mathbb{N}$ 
\begin{itemize}
    \item base induttiva: $0\in A$
    \item passo induttivo: $\forall n \in\mathbb{N}, n\in A\Rightarrow succ(n)\in A$, allora $A=\mathbb{N}$
\end{itemize}

\newpage

\textbf{Principio induttivo di prima forma:}\\
Prendiamo una proposizione P(n) e supponiamo che rispetti 2 condizioni:
\begin{itemize}
    \item la base induttiva: P(0) è vera
    \item il passo induttivo: $\forall n \in\mathbb{N}$, P(n) è vera (ipotesi induttiva), allora P(succ(n))
\end{itemize}
Se rispetta queste condizioni allora implica $\forall n\in\mathbb{N}$, P(n)

\textbf{Teorema di ricorsione:} Sia X un insieme, esite una funzione $f:\mathbb{N}\rightarrow X$ t.c.:
\begin{center}
    $f(0) = c$\\
    $f(succ(n))=h(n,f(n))$
\end{center}

\textbf{Addizione:} tramite il teorema di ricorsione definiamo la funzione $m\rightarrow n+m:$
\begin{center}
    $n + 0 = n$\\
    $n+succ(m)=succe(n)+m$
\end{center}

\textbf{Moltiplicazione:} tramite il teorema di ricorsione definiamo la funzione $m\rightarrow nm:$
\begin{center}
    $n\cdot 0 = 0$\\
    $n(m+1)=mn+n$
\end{center}

\textbf{Ordinamento dei naturali:} può essere totale o parziale

\textbf{Ordine parziale:} è una relazione $\mathcal{R}\subset X\times X$ e rispecchia le seguenti proprità:
\begin{itemize}
    \item riflessiva: $x\mathcal{R}x, \forall x\in X$
    \item antisimmetrica: $x\mathcal{R}y e y\mathcal{R} x\Rightarrow x=y, \forall x,y \in X$
    \item transitiva: $x\mathcal{R}y\ e\ y\mathcal{R} z\Rightarrow x\mathcal{R}z, \forall x,y,z \in X$
\end{itemize}

\textbf{Ordinamento totale:} come l'ordinamento parziale, ma con la proprietà aggiunta:
\begin{itemize}
    \item tricotomia: $x\mathcal{R}y\ o\ y\mathcal{R} x\, \forall x,y \in X$
\end{itemize}

\textbf{insiemi ordinati:} se $\mathcal{R}$ è parziale o totale, dirò che $(X,\mathcal{R})$ è 
parzialmente o totalmente ordinato

\textbf{Principio d'induzione shiftato di prima forma:} identico alla prima forma ma la base
invece che 0, parte da $k\leq n$
\begin{itemize}
    \item base induttiva: P(k) è vera
    \item passo induttivo: $\forall n\geq k$, P(n) è vera $\Rightarrow$ P(n+1)
\end{itemize}

\section{third week}
exercises

\newpage
\section{forth week}

\textbf{Insiemi finiti:} Indicando con I$_{n}$ un insieme che va da 0 a n, diremo che l'insieme X è finito
se essite $n\in\mathbb{N}$ t.c. $I_{n}\sim X$. Se non esiste lo definiremo insieme infinito.

\textbf{Teorema di lemma dei cassetti:} Siano X e Y due insiemi rispettivamente $X\sim I_{n}$ e $Y\sim I_{m}$
con $n<m$ allora la funzione $f(x):\ Y\rightarrow X$ non è iniettiva

\textbf{Cardinalità:} Sia X un insieme finito. Definiamo cardinalità n t.c. $I_{n}$ sia equipotente
a X. Definiamo $I_{n}$ come insieme cardinalità associato a X

\textbf{Proposizione:} Sia $A$ insieme finito e $B\subseteq A$, allora $|B|\leq |A|$

\textbf{Osservazione:} Qualsiasi $f(x): \mathbb{N}\rightarrow\mathbb{N}/\{0\} $ è bigiettiva

\textbf{Minimo:} Sia $A$ un insieme e $z\in A$. Se $\forall x \in A, z\leq x$, allora definiremo $z$ come 
\textbf{minimo} di $A$: 
\begin{center}
    $z = min(A)$
\end{center}

\textbf{Buon ordinamento:} Un ordinamento totale è definito \textbf{ben ordinato} se ogni sottoinsieme di 
$(Z,\leq)$ ammette un minimo

\textbf{Assioma di buon ordinamento:} L'ordinamento $(\mathbb{N},\leq)$ è ben ordinato e l'ordinamento $\leq$
è \textbf{usuale} su $\mathbb{N}$ (cioé se $\exists k\ t.c.\ n+k=m$ allora $n\leq m$)

\textbf{Principio di induzione (2$^{a} forma$):} prendiamo una famiglia di preposizioni $P(n)$ e supponiamo
rispetti le 2 condizioni:
\begin{itemize}
    \item la base induttiva: $P(0)$ è vera
    \item il passo induttivo: $\forall n\in\mathbb{N}, \forall k \in\mathbb{N}\ t.c.\ 0\leq k \leq n$,
    $P(k)$ è vera (ipotesi induttiva), allora P(n)
\end{itemize}
Se rispetta questa condizioni allora implica $\forall n\in\mathbb{N},\ P(n)$

\newpage
\textbf{Divisione euclidea:} Siano $n,m\in\mathbb{Z} t.c. m\neq 0 \exists ! q,r\in\mathbb{Z} t.c:$
\begin{center}
    $n=mq+r$\\$0\leq n <|m|$
\end{center}
(si definiscono $q$ quoziente e $r$ resto della divisione di $n$ per $m$)

\newpage
\section{fifth week}

\textbf{Rappresentabilità:} Sia $b\in\mathbb{N}$, diremo che $n\in\mathbb{N}$ è rappresentabile in base
$b$ se esistono $k\in\mathbb{N}$ e $\varepsilon{0},\varepsilon{1},...,\varepsilon{k}\in I_{b}\ t.c:$
\begin{center}
    \ \ \ \ \ \ \ \ \ \ \ \ \ \ $n = \sum_{i=0}^{k}\ \varepsilon_{i}b^{i}$\ \ \ \ \ \ \ \ con $I_{b}=\{0,1,...,b-1\}$
\end{center} 

\textbf{Teorema della rappresentazione dei naturali in base arbitraria:} Sia $b\in\mathbb{N}, b\geq 2$, allora
$\forall n\in\mathbb{N},\ n$ è rappresentabile in base $b$ in maniera univoca

\textbf{Divisibilità:} Dati $n,m\in\mathbb{Z}$ si dice che $n$ è \textbf{divisore} di $m$ (o $m$ è multiplo
di $n$) se $\exists k\in\mathbb{Z}\ t.c.\ m=nk$ e scriveremo $n|m$

\textbf{Proprità della divisibilità:}
\begin{itemize}
    \item se $n|m$ e $m|q$ allora $n|q$
    \item se $n|m$ e $m|n$ allora $n=\pm m$
\end{itemize}

\textbf{Massimo Comune Divisore:} Dati $m,n\in\mathbb{Z}$ si dice $d\in\mathbb{Z},\ d>0$ massimo comune divisore
se:
\begin{itemize}
    \item $d|n\ e\ d|m$
    \item $\exists c\in\mathbb{Z}\ t.c.\ c|n\ c|m\ c|d$
\end{itemize} 

proposizione: se $d$ e $d^{I}$ sono mcd tra $m$ e $n$ allora $d=d^{I}$

\textbf{Teorema:} dati $n,m\in\mathbb{Z} \neq 0$, esiste mcd unico indicato con $(n,m)$

\textbf{Lemma utile:} dati $n,m,c\in\mathbb{Z} \neq 0$ e $c|n\ c|m$, allora $\forall x,y\in\mathbb{Z}$ vale:
\begin{center}
    $c|xn+ym$
\end{center}

Corollario: Siano $n,m\in\mathbb{Z} \neq 0$ se sia $d:=(n,m)$ allora esistono $x,y\in\mathbb{Z}\ t.c:$
\begin{center}
    $d=xn+ym$
\end{center}

\textbf{Numeri coprimi:} dati $n,m\in\mathbb{Z}$, si dicono coprimi fra di loro se $(n,m)=1$

proposizione: sia $d=n,m$ allora $(\frac{n}{d},\frac{m}{d})=1$

\textbf{Algoritmo di Euclide:}
\begin{center}
    \begin{tabular}{ c c c }
        
        $n=q_{1}m+r_{1}$ & & $r_{1}=n-q_{1}m$\\
        $m=q_{2}r_{1}+r_{2}$ & & $r_{2}=m-q_{2}r_{1}$\\
        $r_{1}=q_{3}r_{2}+r_{3}$ & &$r_{3}=r_{1}-q_{3}r_{2}$\\
        $\cdot$ & &$\cdot$\\
        $\cdot$ & $\Rightarrow$ & $\cdot$\\
        $\cdot$ & & $\cdot$\\
        $r_{k-3}=q_{k-1}r_{k-2}+r_{k-1}$ & &$r_{k-1}=r_{k-3}-q_{k-1}r_{k-2}$\\
        $r_{k-2}=q_{k}r_{k-1}+r_{k}$ & & $r_{k}=r_{k-2}-q_{k}r_{k-1}$\\
        $r_{k-1}=q_{k+1}r_{k}+0$ & & $= xn+ym$
        
    \end{tabular}
\end{center}

Es:
\begin{center}
    (48,28)

    \begin{tabular}{ c c c }
        $48=28\cdot 1 + 20$ & &\\
        $28=20\cdot 1 + 8$ && $4=20-2\cdot 8$\\
        $20=8\cdot 2 + 4$ &$\Rightarrow$& $8=28-20\cdot 1$\\
        $8=\underline{4}\cdot 2 + 0$ &&$20 = 48-28\cdot 1$\\
        $MDC = 4$&&\\
    \end{tabular}

    $\Downarrow$\\
    $4=20-2(28-20\cdot 1)=3\cdot 20 - 2\cdot 28$\\
    $4=3(48-28\cdot 1)-2\cdot 28$\\
    $=\underline{3\cdot 48 - 5\cdot 28}$

\end{center}

\newpage
\section{Sixth week}

\textbf{Proprietà dei coprimi:} Siano $n,m,q\in\mathbb{Z}$ e $n$ o $m\neq 0$ e $(n,m)=1$:
\begin{itemize}
    \item Se $n|mq$ allora $n|q$
    \item Se $n|q$ e $m|q$ allora $nm|q$
\end{itemize} 

\textbf{Numeri primi:} $p\in\mathbb{Z}$ si dice \textbf{primo} se $p\geq 2$ e i suoi divisori sono quelli banali 
($\pm 1|p,\ \pm p|p$). $p$ è primo se $\forall n,m$ e $p|nm$ allora $p|n\bigvee p|m$

\textbf{Minimo Comune Multiplo:} dati $n,m\in\mathbb{Z}$ si dice $M$ minimo comune multiplo di $n$ e $m$ se:
\begin{itemize}
    \item $n|M$ e $m|M$
    \item $\exists c\ t.c.\ n|c, m|c, M|c$
\end{itemize}

\textbf{Unicità mcm:} dati $n,m\in\mathbb{Z}$ e $M,M^{1}$ sono mcm di $n$ e $m$, allora $M=M^{1}$

\textbf{Denotazione mcm:} mcm di $n$ e  $m$ si scrive $[n,m]$

\textbf{Teorema d'esistenza:} siano $n,m\in\mathbb{Z}$ allora $\exists [n,m]$ e se $n\bigvee m\neq 0$ allora:
\begin{center}
    $[n,m] = \frac{nm}{(n,m)}$
\end{center}

\textbf{Teorema fondamentale dell'aritmetica:} $\forall n\in\mathbb{N},n\geq 2$, $n$ è uguale a un prodotto di numeri
primi, anche ripetuti:
\begin{center}
    $n=p_{1}\cdot p_{2}\cdot p_{3}\cdot ... \cdot p_{k}$
\end{center}
La fattorizzazione di questo prodotto è univoca

Corollario: i numeri primi sono infiniti

\textbf{Congruenza:} dati $a,b\in\mathbb{Z}$ diremo che $a$ è  congruo a $b$ modulo $n$ ($a\equiv b\ mod\ n$) se
\begin{center}
    $n|a-b$
\end{center} 

\textbf{Proprietà congruenza:} 
\begin{itemize}
    \item riflessiva: $a\equiv a\ mod\ n\ \ \ \forall a,n\in\mathbb{Z}$
    \item simmetrica: $a\equiv b\ mod\ n$ allora $b\equiv a\ mod\ n\ \ \ \forall a,b,n\in\mathbb{Z}$
    \item transitiva: $a\equiv b\ mod\ n$ e $b\equiv c\ mod\ n$ allora $a\equiv c\ mod\ n\ \ \ 
    \forall a,b,c,n\in\mathbb{Z}$
\end{itemize}

\textbf{equivalenza:} una relazione $\mathcal{R}$ binaria su l'insieme $X$ si dice relazione d'equivalenza su $X$ se:
\begin{itemize}
    \item è riflessiva: $\forall x\in X,\ x\mathcal{R}x$
    \item è simmetrica: $\forall x,y\in X,\ x\mathcal{R}y$ allora $y\mathcal{R}x$
    \item è transitiva: $\forall x,y,z\in X,\ x\mathcal{R}y$ e $y\mathcal{R}z$ allora $x\mathcal{R}z$
\end{itemize}

\newpage
\section{seventh week}

\textbf{Classi d'equivalenza:} sia $X$, $x\in X$ e $\sim$ una relazione d'equivalenza su $X$. Chiameremo classe
d'equivalenza di $x$ in $X$ rispetto a $\sim$ il sottoinsieme di $X$ i quali elementi $y$ sono equivalenti a $x$:
\begin{center}
    $[x]_{\sim}=\{y\in x| y\sim x\}$
\end{center}

\textbf{Insieme quoziente:} chiameremo insieme quoziente di $X$ modulo $\sim$ l'insieme delle classi d'equivalenza
contenute in $X$:
\begin{center}
    $X/\sim =\{ y\in\ x| y\sim x\}$
\end{center}

\textbf{Proprietà classi d'equivalenza:}
\begin{itemize}
    \item $\forall x\in X,\ x\in [x]$
    \item $\forall x,y\in X,\ [x]=[y] \Leftrightarrow x\sim y$
    \item $\forall x,y\in X,\ [x]\cap [y]\neq 0\Rightarrow [x]=[y]$
\end{itemize}

\textbf{Classi di congruenza:} Dati $a,n\in\mathbb{Z}$ definiamo la classe di congruenza di $a$ modulo
$n$ l'insieme delle $x$ congruenti ad $a\ mod\ n$:
\begin{center}
    $[a]_{n}=\{x\in\mathbb{Z}|x\equiv a\ mod\ n\}$
\end{center}
Indicheremo l'insieme quoziente $\mathbb{Z}\ mod\ \sim_{n}$ come $\mathbb{Z}/_{n}\mathbb{Z}$ e ha 
come elementi le classi di congruenza $[a]_{n}$ che appartengono alle partizioni di $\mathbb{Z}$
($2^{\mathbb{Z}}$), quindi:
\begin{center}
    $[a]_{n}=\{a+kn|k\in\mathbb{Z}\}$
\end{center}
Es:\\ $\mathbb{Z}/_{3}\mathbb{Z}=\{[0]_{3},[1]_{3},[2]_{3}\}$

\newpage

\textbf{Prop:} Sia $a\in\mathbb{Z}$ e sia $r$ il resto di $\frac{a}{n}$, allora $a\equiv r\ (mod\ n)$, oppure:
\begin{center}
    $[a]_{n}=[r]_{n}$
\end{center}

\textbf{Criterio di divisibilità:} dati $a,n\in\mathbb{Z}$ con $n\neq 0$, diremo che $a$ è multiplo di $n$ se:
\begin{center}
    $[a]_{n}=[0]_{n}$
\end{center}

\textbf{Notazione:} dato $a\in\mathbb{Z}$ e $x\in [a]_{n}$ ($[a]_{n}=[x]_{n}$), diremo che $x$ è \textbf{rappresentante
della classe} $[a]_{n}$. Se $x$ è di tipo resto, allora $x$ è \textbf{rappresentante canonico}

gli elementi di $\mathbb{Z}/_{n}\mathbb{Z}$ si chiamano \textbf{classi di resto} modulo $n$

\textbf{Struttura algebrica:} esistono due operazioni di somma e moltiplicazione tra insiemi quozienti:
\begin{itemize}
    \item Somma: $[a]_{n}+[b]_{n}=[a+b]_{n}$
    \item Moltiplicazione: $[a]_{n}\cdot [b]_{n}=[a\cdot b]_{n}$
\end{itemize}

\textbf{Prop:} dati $a,a^{1},b,b^{1}\in\mathbb{Z}$ tc $[a]_{n}=[a^{I}]_{n}$ e $[b]_{n}=[b^{I}]_{n}$ allora:
\begin{itemize}
    \item Somma: $[a+b]_{n}=[a^{I}+b^{I}]_{n}$
    \item Moltiplicazione: $[a\cdot b]_{n}=[a^{I}\cdot b^{I}]_{n}$
\end{itemize}

Oss: Sia $a\in\mathbb{Z},\ m\in\mathbb{N},\ m>0$. Allora:
\begin{center}
    $[a]_{n}^{m}=[a_{1}]_{n}\cdot[a_{2}]_{n}\cdot ... \cdot[a_{m}]_{n}\cdot = [a^{m}]_{n}$
\end{center}

\newpage
\section{eight week}
\textbf{Teorema cinese del resto:} Siano $n,m>0$ e siano $a,b\in\mathbb{Z}$. Consideriamo il seguente
sistema di congruenze:
\begin{center}
    \begin{tabular}{ c c c }
        $\begin{cases}
            x\in\mathbb{Z}\\
            x\equiv a(mod\ n)\\
            x\equiv b(mod\ m)
        \end{cases}$
        & o &
        $\begin{cases}
            x\in\mathbb{Z}\\
            [x]_{n}=[a]_{n}\\
            [x]_{m}=[b]_{m}
        \end{cases}$
    \end{tabular}
\end{center}

Sia $S$ l'insieme delle soluzioni dei precedenti Sistemi
\begin{center}
    $S=\langle x\in\mathbb{Z}|x\equiv a(mod\ n)\ e\ x\equiv b(mod\ m) \rangle$
\end{center}
Il precedente sistema è \textbf{compatile} (ammette soluzioni) se e soltanto se:
\begin{center}
    $(n,m)|a-b$
\end{center}
Se $S\neq\varnothing$ e $c\in S$, allora $S=[c]_{[n,m]}\in\mathbb{Z}=\langle c+k_{[n,m]}\in\mathbb{Z}
|k\in\mathbb{Z}\rangle$

Es:

$
\begin{cases}
 x\equiv 9(mod\ 162)\\
 x\equiv -9(mod\ 114)
\end{cases}
$

1 - Compatibilità
\begin{center}
    \begin{tabular}{ c c c }  
        $(162,114) = 6$ & $\Rightarrow$ & $(162,114)| 9-(-9) = 6|18 = 3$\\
        && $\Rightarrow 9-(-9) = 3(162,114)_{(1)}$
    \end{tabular}
\end{center}

2 - Calcolo di una soluzione\\
Algoritmo di Euclide:
\begin{center}
    \begin{tabular}{c|c c c}
        $162=114+48$ & $48=162-114$ && $=18-(48-2\cdot 18)= 3\cdot 18-48$\\
        $114=2\cdot 48+18$ & $18=114-2\cdot 48$ && $=3(114-2\cdot 48)-48 = 3\cdot 114-7\cdot $$48$\\
        $48=2\cdot 18+12$ & $12=48-2\cdot 18$ & $\Rightarrow$ & $=3\cdot 114-7(162-114)=10\cdot 114-7\cdot 162$\\
        $18=12+6$ & $6=18-12$ && $6 = 10\cdot 114-7\cdot 162$\\
        $12=2\cdot 6+0$&&& $(162,114)=10\cdot 114-7\cdot 162_{(2)}$\\
    \end{tabular}
\end{center}
\newpage
Da (1) e (2) segue che 
\begin{center}
    $9-(-9)=3(162,114)=3(10\cdot 114-7\cdot 162)$\\
    $9-(-9)= 30\cdot 114-21\cdot 162_{(3)}$

    $9+21\cdot 162=-9+30\cdot 114 \Rightarrow 3411$
\end{center}
$c=3411$ è una soluzione del sistema

3 - Calcolo di S\\
Teorema cinese del resto:
\begin{center}
    $S=[c]_{[162,114]}=[3411]_{[162,114]}$

    \begin{tabular}{c c c}
        $[162,114]=\frac{162\cdot 114}{(162,114)}=3078$ &$\Rightarrow$ & $S=[3411]_{[3078]}=[333]_{[3078]}$
    \end{tabular}
    
$\Rightarrow S=\langle 333+3078k\in\mathbb{Z}|k\in\mathbb{Z}\rangle$
\end{center}
Bonus:\\
Esiste soluzione di $S$ divisibile da 17?

metodo 1
\begin{center}
    $\begin{cases}
        x\equiv 333(mod\ 3078)\\
        x\equiv 0(mod\ 17)
    \end{cases}$

    $(3078,17)|333-0$

    $1|333$
\end{center}
è divisibile quindi accetta soluzione

metodo 2
\begin{center}
    $[333+3078k]_{17}=[333]_{17}+[3078]_{17}[k]_{17}$\\
    $[10]_{17}+[1]_{17}[k]_[17]=[10+k]_{17}$\\
    $\Rightarrow k=7$
\end{center}

\newpage
\section{neinth week}
\textbf{Elementi invertibili modulo n:}
Siano $a,n\in\mathbb{Z}$ con $n>0$. Diremo che $a$ è invertibile modulo $n$ o equivalentemente che $[a]_{n}$ è
invertibile in $\mathbb{Z}/_{n}\mathbb{Z}$ se esiste $x\in\mathbb{Z}$ tc:
\begin{center}
    $ax\equiv 1(mod\ n) \Leftrightarrow [a]_{n}[x]_{n}=[1]_{n}$
\end{center}
In questo caso diremo che $x$ è un'inversa di $a(mod\ n)$ e $[x]_{n}$ è una classe inversa di $[a]_{n}$ in 
$\mathbb[Z]/_{n}\mathbb{Z}$

\textbf{Lemma:} Supponiamo che $a$ sia invertibile modulo $n$, ovver $[a]_{n}$ sia invertibile in
$\mathbb[Z]/_{n}\mathbb{Z}$. Allora esiste un unico $[x]_{n}\in\mathbb[Z]/_{n}\mathbb{Z}$ tale che:
\begin{center}
    $[a]_{n}[x]_{n}=[x]_{n}[a]_{n}=[1]_{n}$
\end{center}
Equivalentemente $[x]_{n}$ è l'unica classe inversa di $[a]_{n}$ in $\mathbb[Z]/_{n}\mathbb{Z}$.
Tale classe $[x]_{n}$ viene detta inversa e viene indicata con il simbolo $[a]_{n}^{-1}$

\textbf{Prop:} $a\in\mathbb{Z}$ è invertibile $mod\ n\Leftrightarrow(a,n)=1$, in questo caso esiste $x,y\in\mathbb{Z}$
tali che:
\begin{center}
    $xa+yn=1$\\
    $(Algoritmo\ di\ euclide)$
\end{center}
Allora
\begin{center}
    $[a]_{n}^{-1}=[x]_{n}$
\end{center}

Es:\\
$11\ inv(mod\ 30)$

$(11,30) = 1 \Rightarrow \exists[11]_{30}^{-1}$

alg. euclide:\\
$1= 11\cdot 11 + (-4)30$

$[1]_{30}=[(11)(11)+(-4)(30)]=[11]_{30}[11]_{30}+[-4]_{30}[0]_{30}=\underline{[11]_{30}}[11]_{30} \Rightarrow 
[11]_{30}^{-1}=[11]_{30}$

\textbf{Def:} Dato $n\in\mathbb{Z}, n>0$, indichiamo con $(\mathbb{Z}/_{n}\mathbb{Z})$* il sottoinsieme di 
\marginpar{invertibili, cioé mcd è uguale a 1}
$\mathbb{Z}/_{n}\mathbb{Z}$ formato da tutti gli interi modulo $n$ invertibili

\textbf{Prop:} Sia $p$ numero primo, allora vale:
\begin{center}
    $(\mathbb{Z}/_{n}\mathbb{Z})$*$=\{[1]_{p},[2]_{p},...,[p-1]_{p}\}=\mathbb{Z}/_{p}\mathbb{Z}\backslash \{[0]_{p}\}$
\end{center}

\newpage
Prop: Sia $a\in\mathbb{Z}$ e $r$ sia il resto di $a/n$, allora:
\begin{center}
    $a\equiv n(mod\ n)$

    oppure

    $[a]_n=[r]_{n}$
\end{center}

\textbf{Criterio di divisibilità:} dati $a,n\in\mathbb{Z}$ con $n\neq 0$, diremo che $a$ è multiplo di
$n$ se:
\begin{center}
    $[a]_{n}=[0]_{n}$
\end{center}

notazione: dato $a\in\mathbb{Z}$ e $x\in [a]_{n}\ ([a]_{n}=[x]_{n})$, diremo che $x$ è \textit{rappresentante
della classe} $[a]_{n}$. Se $x$ è di tipo resto, allora $x$ è \textit{rappresentante canonico}

Gli elementi di $\mathbb{Z}/_{n}\mathbb{Z}$ si chiamano \textit{classi di resto modulo $n$}

\textbf{Struttura algebrica:} Esistono due operazioni di somma e moltiplicazione tra insiemi quozienti:\\
Somma: $[a]_{n}+[b]_{n}=[a+b]_{n}$\\
Moltiplicazione: $[a]_{n}\cdot[b]_{n}=[ab]_{n}$

Prop: dati $a,a^{I},b,b^{I}\in\mathbb{Z}$ tali che $[a]_{n}=[a^{I}]_{n}$ e $[b]_{n}=[b^{I}]_{n}$, allora:
\begin{itemize}
    \item $[a+b]_{n}=[a^{I}+b^{I}]_{n}$
    \item $[ab]_{n}=[a^{I}b^{I}]_{n}$
\end{itemize}

Oss: Sia $a\in\mathbb{Z},\ m\in\mathbb{N},\ m>0$. Allora:
\begin{center}
    $[a]_{n}^{m}=[a^{m}]_{n}$
\end{center}

\newpage
\section{tenth week}
\textbf{Il teorema di Fermat-Eulero}\\Definiamo la funzione $\phi:\mathbb{N}/\{0\}\rightarrow\mathbb{N}$, detta
\textit{funzione phi di eulero}, ponendo:
\begin{center}
    $\phi(n):=|\{a\in\mathbb{Z}\mid \leq a\leq n,\ (a,n)=1\}|\qquad\forall n\in\mathbb{N}/\{0\}$
\end{center}

Oss: la funzione $\phi$ è moltiplicativa sulle coppie coprime:
\begin{center}
    $\phi(n\cdot m)=\phi(n)\cdot\phi(m)\qquad \forall n,m\in\mathbb{N}/\{0\}\ tc\ (n,m)=1$
\end{center}

Sia $p$ un numero primo e sia $m\in\mathbb{N}/\{0\}$. Considero $n=p^{m}$, allora $\phi(n)=\phi(p^{m})$ che vale:
\begin{center}
    $\phi(p^{m})=p^{m}-p^{m-1}\qquad \forall p$ primo e $\forall m\in\mathbb{N}/\{0\}$
\end{center}

\textbf{Formula generale:} Sia $n\geq 2$ e $n=p_{1}^{m_{1}}\cdot p_{2}^{m_{2}}\cdot...\cdot p_{k}^{m_{k}}$ per 
qualche numero primo $p_{1},p_{2},...,p_{k}$ con $p_{i}\neq p_{j} \forall i\neq j$ e $m_{1},...,m_{k}\in\mathbb{N}
/\{0\}$. Allora:
\begin{center}
    $\phi(n)=\phi(p_{1}^{m_{1}}\cdot ... \cdot p_{k}^{m_{k}})=(p_{1}^{m_{1}}-p_{1}^{m_{1-1}})\cdot ... \cdot
    (p_{k}^{m_{k}}-p_{k}^{m_{k-1}})$
\end{center}

\textbf{Lemma:} Dato $n>0$, vale:
\begin{center}
    $|(\mathbb{Z}/_{n}\mathbb{Z})^{*}|=\phi(n)$
\end{center}

\textbf{Lemma:} Dati $\alpha,\beta\in(\mathbb{Z}/_{n}\mathbb{Z})^{*}$, valgono le seguenti affermazioni:
\begin{itemize}
    \item $\alpha\beta\in(\mathbb{Z}/_{n}\mathbb{Z}),\quad (\alpha\beta)^{-1}=\alpha^{-1}\beta^{-1}$
    \item $\alpha^{-1}\in(\mathbb{Z}/_{n}\mathbb{Z}),\quad (\alpha^{-1})^{-1}=\alpha$
\end{itemize}

\textbf{Teorema:} Sia $n>0$. Per ogni $\alpha\in(\mathbb{Z}/_{n}\mathbb{Z})^{*}$, vale:
\begin{center}
    $\alpha^{\phi(n)}=[1]_{n}$ in $\mathbb{Z}/_{n}\mathbb{Z}$
\end{center}
Equivalentemente, per ogni $a\in\mathbb{Z}$ tale che $(a,n)=1$
\begin{center}
    $a^{\phi(n)}\equiv 1(mod\ n)$
\end{center}

Corollario: Se $p$ è un numero primo e $a\in\mathbb{Z}$ tale che $(p,a)=1$, allora:
\begin{center}
    Con $n=p\Rightarrow a^{p-1}\equiv 1(mod\ p)$
\end{center}

\newpage
\marginpar{09/05/21}
\textbf{Crittografia RSA}\\
Fissiamo  $n>0$. Per ogni $c\in\mathbb{N}/\{0\}$, definiamo la funzione:
\begin{center}
    $P_{c}:(\mathbb{Z}/_{n}\mathbb{Z})^{*}\rightarrow(\mathbb{Z}/_{n}\mathbb{Z})^{*}$\\
    $\alpha\rightarrow\alpha^{c}$
\end{center}
Ovvero $P_{c}(\alpha):= \alpha^{c}\quad\forall\alpha\in(\mathbb{Z}/_{n}\mathbb{Z})^{*}$. La funzione $P_{c}$ è
ben definita, ovvero, se $\alpha$ è una classe di congruenza e $n$ invertibile, allora anche $\alpha^{c}$ è 
invertibile

\textbf{Teorema della crittografia RSA}\\ Sia $c\in\mathbb{N}/\{0\}$ tale che $(c,\phi(n))=1$ e sia $d\in\mathbb{N}
/\{0\}$ un inverso di $c$ modulo $\phi(n)$ (ovver $d>0$ e $d\in[c]^{-1}_{\phi(n)}$), allora $P_{c}$ è una funzione
invertibile e vale $P_{c}^{-1}=P_{d}$
\begin{center}
    \begin{tabular}{c l}
        $P_{d}=P_{c}^{-1}\Leftrightarrow$ & $p_{d}(P_{c}(\alpha))=\alpha\qquad\forall\alpha\in(\mathbb{Z}/_{n}
        \mathbb{Z})^{*}$\\
        & $p_{c}(P_{d}(\beta))=\beta\qquad\forall\beta\in(\mathbb{Z}/_{n}\mathbb{Z})^{*}$\\
    \end{tabular}
\end{center}

Corollario: Siano $a,c\in\mathbb{Z}$ tale che $(a,n)=1$ e $c>0$. Considero la seguente congruenza 
in $x\in\mathbb{Z}$
\begin{center}
    $x^{c}\equiv a (mod\ n)$
\end{center}
Sia $S$ l'insieme delle soluzioni della precedente congruenza, ovvero:
\begin{center}
    $S:=\{x\in\mathbb{Z}\mid x^{c}\equiv a(mod\ n)\}$
\end{center}
allora se $(c,\phi(n))=1$ e $d>0$ con $d\in[c]^{-1}_{\phi(n)}$, Allora
\begin{center}
    $S=[a^{d}]_{n}=\{a^{d}+kn\in\mathbb{Z}\mid k\in\mathbb{Z}\}$
\end{center}

\textbf{Crittografia a chiave pubblica}\\Supponiamo che A voglia comunicare con B mediante RSA:\\
B pubblica $c,n\in\mathbb{Z}/\{0\}$, $c$ chiave di codifica e $n$ modulo tale che $(c,\phi(n))=1$. A userà
l'alfabeto $(\mathbb{Z}/_{n}\mathbb{Z})^{*}$. Se A comunica $\alpha\in(\mathbb{Z}/_{n}\mathbb{Z})^{*}$, allora 
calcola $\alpha^{c}$ e lo invierà. Allora B $(c,\phi(n))\rightarrow d>0, d\in[c]^{-1}_{\phi(n)}$ con $d$ 
chiave di decifratura. Quindi $\beta\rightarrow\beta^{d}=\alpha$


\newpage
\section{Eleventh week}
\textbf{Grafi:} Dato un insieme $V$, indichiamo $\binom{V}{2}$ l'insieme i cui elementi sono tutti sottoinsiemi di 
$V$ con 2 elementi, ovvero:
\begin{center}
    $\binom{V}{2}:=\{A\in 2^{V}\mid |A|=2\}$
\end{center}
Vale la formula:
\begin{center}
    $|\binom{V}{2}|=\binom{|V|}{2}=\frac{|V|!}{2!(|V|-2)!}=\frac{|V|(|V|-1)}{2}$
\end{center}

\textbf{Def:} Un grafo $G$ è una coppia $(V,E)$, dove $V$ è un'insieme non vuoto detto \textit{insieme
dei vertici di $G$} e $E$ è un sottoinsieme di $\binom{V}{2}$ detto \textit{insieme dei lati di $G$}.\\
Se $G=(V,E)$ è un grafo ed $e=\{v_{1},v_{2}\}\in E$, cioé un lato di $G$, allora diciamo che $v_{1}$ e
$v_{2}$ sono degli estremi di $e$

Se $G$ è un grafo, allora $V(G)$ indica l'insieme dei vertici e $E(G)$ l'insieme dei lati di $G$. Se 
$G=(V,E)$ è un grafo ed $e=\{v_{1},v_{2}\}\in E$, cioé un lato di $G$, allora diciamo che $v_{1}$ e
$v_{2}$ sono gli \textit{estremi} di $e$ ed anche che $e$ congiunge $v_{1}$ e $v_{2}$

\textbf{Esempi notevoli}
\begin{itemize}
    \item Per ogni $n\in\mathbb{N}$, definiamo il \textit{cammino} $P_{n}$ di lunghezza $n$ come il
    seguente grafo 
    \begin{center}
        $V(P_{n})=\{0,1,...,n\}\ E(P_{n}):= \varnothing $ se $n=0$\\
        $E(P_{n}):=\{\{i,i+1\}\in\binom{V(P_{n})}{2}\}$ 
    \end{center}
    \item $P_{\infty}$ il cammino infinito
    \item Per ogni $n\in\mathbb{N}$ con $n\geq 3$, il \textit{ciclo} di lunghezza $n$ è definito:
    \begin{center}
        $V(C_{n})=\{1,2,...,n\}\ E(C_{n})=\{\{i,i+1\}\in\binom{V(C_{n})}{2}\}\cup\{\{1,n\}\}$
    \end{center}
    \item Per ogni $n\in\mathbb{N},\ n\geq 1$, il \textit{grafo completo} di $n$ vertici, denotato con
    $k_{n}$, è definito:
    \begin{center}
        $V(k_{n})_:=\{1,2,...,n\},\ E(kn):=\binom{V(kn)}{2}$
    \end{center}
\end{itemize}

\newpage
\textbf{Sottografi e sottografi indotti}\\ Siano $G=(E,V)$ e $G^{I}=(E^{I},V^{I})$ due grafi. Diremo 
che $G^{I}$ è un sotto grafo di $G$ se $V^{I}\subset V$ e $E^{I}\subset E$

Se $G^{I}$ è sottografo di $G$ vale:
\begin{center}
    $E^{I}=\{e\in E\mid e=\{v_{1},v_{2}\}, v_{1}\in V^{I},v_{2}\in V^{I}\}$
\end{center}
allora $G^{I}$ si dice \textit{sotto grafo} di $G$ \textit{indotto} da $V^{I}$ e si indica con il
simbolo $G[V^{I}]$


\marginpar{13/05/21}
\textbf{Morfismi}\\ Siano $G=(V,E)$ e $G^{I}=(V^{I},E^{I})$ due grafi, e $f:V\rightarrow V^{I}$ 
una funzione iniettiva. Allora si dice morfismo da $G$ a $G^{I}$ se vale:
\begin{center}
    $\forall v_{1},v_{2}\in V,\ \{v_{1},v_{2}\}\in E\Rightarrow \{f(v_{1}),f(v_{2})\}\in E^{I}$
\end{center}
Se $f:V\rightarrow V^{I}$ è un morfismo da $G$ a $G^{I}$, allora scriveremo $f:G\rightarrow G^{I}$

Oss: Siano $G=(V,E)$ e $G^{I}=(V^{I},E^{I})$ due grafi, sia $f:G\rightarrow G^{I}$. Per ogni 
$e=\{v_{1},v_{2}\}\in E$, allora:
\begin{center}
    $f(e)=\{f(v_{1},v_{2})\}\in\binom{V^{I}}{2}$
\end{center}
Definiamo $f(E):=\{f(e)\in\binom{V^{I}}{2}\mid e\in E\}$, segue che:
\begin{center}
    $f(E)\subset E^{I}$
\end{center}
Dunque $f$ è un morfismo solo se $f(E)\subset E^{I}$

\textbf{Isomorfinsmo}\\ Diciamo che $f$ è un isomorfinsmo da $G$ in $G^{I}$ se:
\begin{itemize}
    \item $f$ è bigiettiva
    \item $f$ è morfismo da $G$ in $G^{I}$
    \item $f^{-1}:V(G)\rightarrow V(G)$ è un morfismo da $G^{I}$ in $G$. Se esiste un isomorfismo, allora
    $G$ si dice isomorfo a $G^{I}\Rightarrow G\cong G^{I}$
\end{itemize}

Prop: Siano $G$, $G^{I}$ due grafi e $f:V\rightarrow V^{I}$ una funzione. $f$ è isomorfismo da $G$ in $G^{I}$ 
se e solo se:
\begin{itemize}
    \item $f$ è bigiettiva
    \item $f(E)=E^{I}$, ovvero $\forall e\in\binom{V}{2},\ e\in E\Leftrightarrow f(e)\in E^{I}$
\end{itemize}

\newpage
\textbf{Passeggiate, cammini e cicli}\\ Sia $G$ una successione finita ordinata di vertici di $G$. Allora si dice:
\begin{itemize}
    \item Passeggiata in $G$, se $n=0$ oppure $n\geq 1$ e $\{v_{i},v_{i+1}\}\in E\quad\forall i\in\{0,1,...,n-1\}$
    \item Cammino in $G$, se è una passeggiata in $G$ e $v_{i}\neq v_{j}\quad\forall i,j\in\{0,1,...,n\}$
    \item Ciclo in $G$ se è una passeggiata in $G$, $v_{0}=v_{n}$ e $n\geq 3,\ v_{i}\neq v_{j}$
\end{itemize}
Se $(v_{0},v_{1},...,v_{n})$ è una passeggiata in $G$, allora $n$ è detto lunghezza, $n=l(G)$

\textbf{Def}\\ Sia $G$ un grafo e siano $v,w\in V$. Sono \textit{congiungibili} in $G$ con passeggiata se 
esiste una passeggiata in $G$ della seguente forma: $(v_{0},v_{1},...,v_{n})$

\newpage
\section{Dodicesima settimana}
\textbf{Congiungibilità}\\ Sia $G=(V,E)$ e siano $v,w\in V$. Diciamo che $v$ e $w$ sono congiungibili con un cammino
se esiste un cammino $(v_{0},v_{1},...,v_{n})$ tale che $v_{0}=v$ e $v_{n}=w$

\textbf{Prop}: Sia $G=(V,E)$ e siano $v,w\in V$. Allora $v$ e $w$ sono congiungibili con un cammino se e soltanto 
se lo sono con una passeggiata

Oss: Dato un grafo $G=(V,E)$ e $v,w\in V$ diciamo che $v$ e $w$ sono congiungibili se lo sono per cammini o
passeggiate

\textbf{Prop:} Sia $G=(V,E)$ e sia $\sim$ la relazione binaria su $V$ indetta dalla nozione di congiungibilità
in $G$: $\sim\in\mathcal{P}(V\times V)$ è definita ponendo $v\sim w$ se $v$ è congiungibile a $w$ in $G$.
Allora $\sim$ è una relazione di equivalenza in $V$. Allora $\sim$ è una relazione di equivalenza in $V$

\textbf{Def:}Sia $G=(V,E)$ e  sia $\sim$ la relazione di congiungibilità su $V$, indichiamo con $\{V_{i}\}_{i\in I}$
l'insieme di tute le $\sim$ classi d'equivalenza.I sotto grafi $\{G[V_{i}]\}_{i\in I}$ indotti  da $G$ su $V_{i}$
si dicono \textit{componenti connesse} di $G$

\textbf{Grafi connessi}\\Un grafo si dice \textit{connesso} se possiede una sola componente connessa. Altrimenti
si definisce \textit{sconnesso}

Oss:
\begin{itemize}
    \item Sia $G$ un grafo, allora $G$ è connesso se e solo se ogni coppia di vertici di $G$ è congiungibile
    in $G$
    \item Ogni componente connessa di $G^{I}$ di $G$ è un grafo connesso
\end{itemize}

\textbf{Prop:} Siano $G$ e $G^{I}$ due grafi e sia $f:G\to G^{I}$ un morfismo. Valgono:
\begin{itemize}
    \item se $v,w\in V(G)$ tale che $v$ è raggiungibile a $w$ in $G$, allora $f(w)$ e $f(v)$ sono 
    congiungibili in $G^{I}$
    \item Se $f$ è un isomorfinsmo, allora $v\sim w$ in $G\Leftrightarrow f(v)\sim f(w)$ in $G^{I}$
\end{itemize}

Corollario: Siano $G$ e $G^{I}$ due grafi isomorfi, siano $\{G_{i}\}_{i\in I}$ le componenti connesse di $G$ e 
$\{G^{I}_{j}\}_{j\in I}$ le componenti connesse di $G^{I}$. Allora $G$ e $G^{I}$ hanno lo stesso numero di componenti
connesse e tali componenti sono 2 a 2 isomorfe. Più precisamente, $\exists\varphi:I\to J$ una bigezione talche che 
$G_{i}\cong G^{I}_{\varphi(i)}\quad\forall i\in I$

Corollario: Due grafi isomorfi sono entrambi connessi o non connessi

\newpage
\textbf{Relazione fondamentale tra gradi dei vertici e numero dei lati di un grafo finito}

\textbf{Def:}Un grafo $G$ è detto finito se ha un numero finito di vertici

Oss: Un grafo finito possiede anche un numero finito di lati.\\Viceversa è falso, esistono grafi con infiniti vertici
e finiti lati.

\textbf{Def:} Sia $G$ un grafo finito e sia $v\in V$. Definiamo il grado $deg_{G}(v)$ di $v$ in $G$ ponendo:
\begin{center}
    $deg_{G}(v):=|\{e\in E\mid v\in E\}|$

    \small{(o numero di lati che escono da $v$)}
\end{center}

\marginpar{20/05/21}
\textbf{Prop:} Sia $G=(V,E)$ un grafo finito. Allora:
\begin{center}
    $\sum_{v\in V}\ deg_{G}(v)=2|E|$
\end{center}

\textbf{lemma delle strette di mano:} In un grafo finito, il numero di vertici di grado dispari è pari

\textbf{Def:} Sia $G=(V,E)$ un grafo finito con $n$ vertici, definiamo con \textit{score} di $G$, con il simbolo
$score(G)$, come la n-upla di interi equali ai gradi dei vertici di $G$.\\ Diremo che lo score è in \textit{forma
canonica} se la successione è ordinata in modo non decrescente

\textbf{Prop:} Siano $G$ e $G^{I}$ due grafi isomorfi, vale:
\begin{center}
    $score(G)=score(G^{I})$
\end{center}
Il contrario è falso, esistono grafi non isomorfi ma con score pari.

\textbf{Grafi 2-connessi e grafi di Hamilton}

\textbf{Def:} Sia $G=(V,E)$ un grafo finito con almeno 2 vertici e sia $v\in V$. definitamo $G-v$ il grafico
ottenuto da $G$ rimuovendo $v$, ponendo:
\begin{center}
    $V(G-v):=V/\{v\},\quad E(G-v):=\{e\in E, v\notin V\}$
\end{center}

\textbf{Def:} un grafo $G$ si dice \textit{2-connesso} se ha almeno 3 vertici e $\forall v\in V(G)$, $G-v$ è connesso

\textbf{Lemma:} ogni grafo 2-connesso è anche connesso. il contrario non vale.

\textbf{Def:} Sia $G$ un grafo. UN ciclo in $G$ che attraversa tutti i vertici di $G$ è detto \textit{ciclo 
Hamiltoniano}. Se $G$ ammette almeno un ciclo Hamiltoniano è  detto \textit{grafo Hamiltoniano}

Oss: Un Hamiltoniano è  sempre un grafo finito e ha almeno 3 vertici

\textbf{Lemma:} Un grafo Hamiltonianoè anche 2-connesso


\newpage
\section{thirteenth week}
\textbf{Foglia:} sia $G=(V,E)$ un grafico e sia $v\in V$. Diciamo che $v$ è una \textit{foglia} di $G$ se 
$deg_{G}(v)=1$

\textbf{Lemma:} Un grafo 2-connesso o hamiltoniano non possiede foglie

\textbf{Lemma:} Siano $G$ e $G^{I}$ due grafi isomorfi. Valgono le seguenti affermazioni:
\begin{itemize}
    \item $G$ è 2-connesso solo se lo è anche $G^{I}$
    \item $G$ è Hamiltoniano solo se lo è anche $G^{I}$
\end{itemize}

\textbf{Note:} per determinare l'isomorfismo di un grafo, possiamo verificare alcune caratteristiche:
\begin{itemize}
    \item $score(G)=score(G^{I})$
    \item $G$ e $G^{I}$ sono entrambi connessi o meno. Il numero di componenti connesse è lo stesso
    \item entrambi sono 2-connessi o meno
    \item entrambi sono Hamiltoniani o meno
    \item hanno lo stesso numero di sottocili
    \item scelto un vertice di $G$ di grado $k$, allora tutti $k$ vertici collegati a $f(v)$ devono avere lo 
    stesso score di quelli collegati a $v$
\end{itemize}
Se tutte queste regole sono rispettate non per forza i due grafi sono isomorfi. Il modo pià accurato per determinare
l'isomorfismo è di generarne uno a mano 

\textbf{Lemma:} Sia $n\in\mathbb{N}$ con $n\geq 1$, allora se $G=(V,E)$ è un grafo con $n$ vertici, vale:
\begin{center}
    $deg_{G}(v)\leq n-1\qquad\forall v\in V$
\end{center}

Corollario: Sia $n\in\mathbb{N}$ con $n\geq 1$ e sua $d=(d_{1},...,d_{n})\in\mathbb{N}^{n}$ ordinato. Se 
$d_{n}>n-1$ allora nessun grafo avrà $d$ come score (ost 1)

Oss: Siano $n,m\in\mathbb{N}/\{0\}$ e  sia $d\in\mathbb{N}^{n+m}$ in forma $(0_{1},...,0_{m},d_{1},...,d_{n})$ 
ordinata. Definiamo $d^{I}$ come $d$ senza gli zero. Anche $d^{I}$ è score del grafo

\textbf{Lemma:} Siano $n,k\in\mathbb{N}/\{0\}$ tale che $k<n$ e $h:=n-k$ e $d$ score in forma $(d_{1},...,d_{h},
n-1_{1},...,n-1_{k})$ ordinata. Se $d_{1}<k$ allora $d$ non è score di un grafo (ost 2)

\marginpar{27/05}
\textbf{Lemma:} Sia $n\in\mathbb{N}$ con $n\geq 3$, sia $d=(d_{1},...,
d_{n})\in\mathbb{N}^{n}$ ordinato e sia $L$:
\begin{center}
    $L:=|\{i\in\{1,..., n-2\}\mid d_{i}\geq 2\}|$
\end{center}
Se $L<d_{n-1}+d_{n}-n$, allora d non è score di un grafo (ost3)

\textbf{Lemma:} Sia $n\in\mathbb{N}/\{0\}$ e $d\in\mathbb{N}^{n}$ un vettore ordinato tale che $d_{1}\leq ...\leq
d_{n}\leq 2$. Vale:
\begin{itemize}
    \item se $d=(0_{1},...,0_{n-1},2)$ oppure $n\geq 2$ e $d=(0_{1},...,0_{n-2},2,2)$, allora non è un grafo
    \item Se $d=(0,...,0)$ allora $d$ è lo score di un grafo con $n$ vertici isolati. Se esiste $m\in\mathbb{N}$
    tale che $n\geq m\geq 3$ e $d=(0_{1},...,0_{n-m},2_{1},...,2_{m})$ allora $d$ è lo score del grafico con $n-m$
    vertici isolati e $m$ vertici in ciclo
\end{itemize}
Se per $k\in\mathbb{N}$ volte lo score è pari a 1 allora il grafo è formato da $k/2$ segmenti

Corollario: Se il numero di $d=1$ è pari, allora $d$ è lo score di un grafo solo se $d$ non ha le forme:
\begin{center}
    $d=(0,...,0,2)\quad$ oppure $\quad d=(0,...,0,2,2)$
\end{center}

\textbf{Teorema dello score}\\
Sia $n\in\mathbb{N}$ con $n\geq 2$ e $d\in\mathbb{N}^{n}$ tale che $d_{1}\leq ...\leq d_{n}\leq n-1$. definiamo il 
vettore ponendo:
\begin{center}
    $d_{i}^{I}
    \begin{cases}
        d_{i} & se\ i<n-d_{n}\\
        d_{i}-1 & se\ i>n-d_{n}
    \end{cases}
    \qquad\forall i\in\{1,...,n-1\}$
\end{center}
$d$ è lo score di un grafo solo se lo è $d^{I}$

\newpage
\section{fourteenth week}
\textbf{Alberi}\\Un grafo si dice albero se è connesso e senza cicli. Una foresta è un grafo senza cicli

\textbf{Teorema:} Sia $T=(V,E)$ un grafo. Le seguenti affermazioni sono equivalenti:
\begin{enumerate}
    \item $T$ è un albero
    \item Per ogni $v,\ v^{I}\in V$, esiste un unico cammino in $T$ che congiunge $v$ a $V^{I}$
    \item $T$ è connesso e, per ogni $e\in E$, il grafo $T-e$ è sconnesso
    \item $T$ non ha cicli e, per ogni $e\in\binom{V}{2}\backslash E$, il grafo $T+e$, definito ponendo $T+e:=(V,
    E\cup\{e\})$, ha almeno un ciclo
    \item $T$ è connesso e soddisfa la seguente formula di eulero:
    \begin{center}
        $|V|-1=|E|$
    \end{center}
\end{enumerate} 

\textbf{Lemma:} Sia $T$ un albero finito avente almeno 2 vertici. Allora $T$ ha almeno 2 foglie

Osservazione: Il precedente lemma non vale se l'albero è infinito

\textbf{Teorema:} La connessione di $T$ non può essere omessa per l'applicazione della formula di eulero

\textbf{Corollario:} Sia $n\in\mathbb{N}\backslash\{0\}$ e sia $d\in\mathbb{N}^{n}$. Allora esiste un albero con 
score $d$ se e solo se:
\begin{center}
    $n-1=\frac{1}{2}(\sum^{n}d_{i})$
\end{center}

\newpage
\section{Esercizi}
\subsection{Dimostrazione per induzione}
Trovo una base dell'induzione (0,1,...) e dimostro l'eguaglianza. Poi applico il passo induttivo (n+1).
Applico l'ipotesi induttiva, cioé faccio valere l'uguaglianza in n. 
\subsection{Sistema di congruenze}
Es:$\begin{cases}
    x\equiv 4 (mod\ 112)\\
    x\equiv 20 (mod\ 50)
\end{cases}$
\begin{enumerate}
    \item Compatibilità: Calcolo l'mcd dei due moduli tramite l'algoritmo di euclide e 
    verifico che mcd divida la differenza tra i due valori. Grazie al teorema cinese del resto
    sapremo che l'insieme delle soluzioni è vuoto o meno. Mi tengo da parte la formula (1) che
    consiste nell'uguaglianza tra la differenza dei due valori e l'mcd dei due moduli per il loro 
    moltiplicando
    \begin{center}
        $(112,50)=2\qquad 2|20-4$\\
        $20-4=8\cdot(112,50)^{(1)}$
    \end{center}
    \item Calcolo di una soluzione: Proseguo con l'algoritmo di euclide e ottengo la formula (2).
    Adesso unisco le formule (1) e (2) ottenendo una soluzione del sistema
    \begin{center}
        $(112,50)=50\cdot 9 - 4\cdot 112^{(2)}$\\
        $20-4=8(9\cdot 50-4\cdot 112\Rightarrow 20-72\cdot(50)=4-32\cdot 112=-3580$
    \end{center}
    \item Calcolo di S: inserisco la soluzione trovata nella classe pari al mcm dei due moduli,
    normalizzo la classe e ottengo l'insieme delle soluzioni
    \begin{center}
        $[112,50]=\frac{122\cdot 50}{(112,50)}=2800$\\
        $[-3500]_{2800}=[2020]_{2800}\Rightarrow\{2020+2800k\in\mathbb{Z}\mid k\in\mathbb{Z}\}$
    \end{center}
\end{enumerate}
\subsection{RSA}
Es: $x^{25}\equiv 8(mod\ 63)$
\begin{enumerate}
    \item Applicabilità: verifico che l'MCD tra valore e modulo e l'MCD tra esponente e phi del modulo 
    diano come risultato 1, se si allora RSA è applicabile
    \begin{center}
        $(8,63)=(2^{3},3^{2}\cdot 7)=1$\\
        $(25,\Phi(63))=(25,36)=(5^{2},2^{2}\cdot 3^{2})=1\qquad \Phi(63)=(3^{2}-3^{1})(7^{1}-7^{0})
        =6\cdot 6=36$
    \end{center}
    \item Calcolo di S: Applico l'algoritmo di Euclide al phi del modulo e l'esponente. Il risultato lo applico
    alla classe di Phi modulo. Applico le trasformazioni e ottengo l'inverso della classe dell'esponente.
    Utilizzo il risultato trovato e lo imposto come d. Risolvo la classe e ottengo la nostra soluzione
    \begin{center}
        algoritmo di euclide di 25 e 36$\rightarrow 1=13\cdot25+(-9)36$\\
        $[1]_{36}=[13]_{36}[25]_{36}\Rightarrow [25]^{-1}_{36}=[13]_{36}$\\
        $[8^{13}]_{63}=[8]_{63}[8^{2}]^{6}_{63}=[8]_{63}\Rightarrow S=\{8+63k\in\mathbb{Z}\mid 
        k\in\mathbb{Z}\}$
    \end{center}
\end{enumerate}
\subsection{Grafi}
\textbf{Ostruzioni}
\begin{itemize}
    \item Ostruzione 1 - Se $d_{n} > n-1$, allora d non è score di un grafo
    \item Ostruzione 2 - Se $d$ è in forma $(d_{1},...,d_{h},n-1_{1},...,n-1_{k})$ e $d_{1}<k$, allora 
    $d$ non è score di un grafo
    \item Ostruzione 3 - Se $d\in\mathbb{N}^{n}$ con $n\geq 3$ e sia $L=|\{i\in\{1,...,n-2\}\mid
    d_{i}\geq 2\}|$. Se $L<d_{n-1}+d_{n}-n$, allora $d$ non è score di un grafo
    \item Ostruzione 4 - Se $d$ possiede un numero dispari di componenti dispari, allora $d$ non è score
    di un grafo
\end{itemize}
\textbf{Teorema dello score}:Applicabile se $d_{n}\leq n-1$

\textbf{Formula di Eulero}: Uno score può essere un albero se rispette l'uguaglianza $|V|-1=|E|$ sapendo
che $|E|=\frac{1}{2}\sum d_{i}$

\textbf{Forzatura alla connessione/sconnessione}: Se un grafo rispetta $|E|<|V|-1$ allora è un grafo 
sconnesso.\\Al contrario se $v_{0}\geq n - v_{n} -1$ allora è un grafo connesso

\end{document}