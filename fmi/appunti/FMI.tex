\documentclass[11pt, letterpaper]{article}
\usepackage[utf8]{inputenc}
\usepackage{fixltx2e}
\usepackage{amsmath}
\usepackage{amsfonts}
\usepackage{multicol}
\usepackage{amssymb}
\usepackage{geometry}
\geometry{margin=1.2in}
\title{Statistica}

\setlength{\parindent}{0em}
\setlength{\parskip}{1em}

\begin{document}
    
\section{First week}
\textbf{2$^{A}$:} insieme delle parti di A $\Rightarrow\ 2^{\wedge\#A}$ = elenco delle parti di A

\textbf{Relazioni:} dati 2 insiemi X e Y, e un sottoinsieme $\mathcal{R}$(X,Y) è detto relazione tra 
X e Y e scriveremo x$\mathcal{R}$y, x,y$\in\mathcal{R}$

\textbf{Funzione:} siano dati X,Y e dia f una relazione tra X e Y, f $\subset X\times Y$.
diremo che f è una funzione da X in Y se vale:
\begin{center}
    $\forall x \in X: \exists ! y \in Y t.c. (x,y)\in f$
\end{center}

\textbf{Dominio:} insieme delle x che vanno in Y

\textbf{Codomidio:} insieme delle y che hanno corrispondenza in X

\textbf{Legge:} proprietà che definisce una relazione da X a Y

\textbf{Insieme di tutte le funzioni:} $Y^{X}$ corrisponde a tutte le funzioni con leggi diverse ma
con stessi insiemi di partenza ed arrivo

\textbf{Funzione identità:} $id_{X}(X) = X$

\textbf{Composizione di funzioni:} $x\rightarrow^{f} y\rightarrow^{g} z \Rightarrow g(f(x)) = z
\Rightarrow gof(x) = z$ 

\textbf{Iniettiva:} ad ogni f(x) corrisponde un solo y\\
\textbf{Surgettiva:} ad ogni y corrisponde un f(x)\\
\textbf{Bigiettiva:} sia iniettiva che suriettiva

\textbf{Inversa:} se f è biettiva, allora esiste $g=f^{-1}$

\newpage
\section{Second week}

\textbf{Sistemi equipotenti:} X e Y sono equipotenti ($X\sim Y$) se hanno la stessa
cardinalità e la funzione $f:X\rightarrow Y$ è bigiettiva (o invertibile)

\textbf{insiemi cardinali:} sono gli insiemi in formato \{0,1,...,n\} equipotenti all'insieme dato, si 
rapprensentano $|A|$ e definiscono una cardinalità pari a n+1

\textbf{TEOREMA:} X e Y sono equipotenti se e solo se i loro insiemi cardinali sono uguali
\begin{center}
    $|X|=|Y|$
\end{center}

\textbf{Numeri naturali:} sono definiti dagli assiomi di Peano:
\begin{itemize}
    \item 0 è un numero naturale
    \item esiste una funzione successivo $\mathbb{N}\rightarrow\mathbb{N}$
    \item succ(n) $\in\mathbb{N}\backslash\{0\}$, cioé il successivo di ogni naturale è diverso da 0
    \item vale principio d'induzione
\end{itemize}

\textbf{Principio d'induzione:} con $A\subset\mathbb{N}$ 
\begin{itemize}
    \item base induttiva: $0\in A$
    \item passo induttivo: $\forall n \in\mathbb{N}, n\in A\Rightarrow succ(n)\in A$, allora $A=\mathbb{N}$
\end{itemize}

\newpage

\textbf{Principio induttivo di prima forma:}\\
Prendiamo una proposizione P(n) e supponiamo che rispetti 2 condizioni:
\begin{itemize}
    \item la base induttiva: P(0) è vera
    \item il passo induttivo: $\forall n \in\mathbb{N}$, P(n) è vera (ipotesi induttiva), allora P(succ(n))
\end{itemize}
Se rispetta queste condizioni allora implica $\forall n\in\mathbb{N}$, P(n)

\textbf{Teorema di ricorsione:} Sia X un insieme, esite una funzione $f:\mathbb{N}\rightarrow X$ t.c.:
\begin{center}
    $f(0) = c$\\
    $f(succ(n))=h(n,f(n))$
\end{center}

\textbf{Addizione:} tramite il teorema di ricorsione definiamo la funzione $m\rightarrow n+m:$
\begin{center}
    $n + 0 = n$\\
    $n+succ(m)=succe(n)+m$
\end{center}

\textbf{Moltiplicazione:} tramite il teorema di ricorsione definiamo la funzione $m\rightarrow nm:$
\begin{center}
    $n\cdot 0 = 0$\\
    $n(m+1)=mn+n$
\end{center}

\textbf{Ordinamento dei naturali:} può essere totale o parziale

\textbf{Ordine parziale:} è una relazione $\mathcal{R}\subset X\times X$ e rispecchia le seguenti proprità:
\begin{itemize}
    \item riflessiva: $x\mathcal{R}x, \forall x\in X$
    \item antisimmetrica: $x\mathcal{R}y e y\mathcal{R} x\Rightarrow x=y, \forall x,y \in X$
    \item transitiva: $x\mathcal{R}y\ e\ y\mathcal{R} z\Rightarrow x\mathcal{R}z, \forall x,y,z \in X$
\end{itemize}

\textbf{Ordinamento totale:} come l'ordinamento parziale, ma con la proprietà aggiunta:
\begin{itemize}
    \item tricotomia: $x\mathcal{R}y\ o\ y\mathcal{R} x\, \forall x,y \in X$
\end{itemize}

\textbf{insiemi ordinati:} se $\mathcal{R}$ è parziale o totale, dirò che $(X,\mathcal{R})$ è 
parzialmente o totalmente ordinato

\textbf{Principio d'induzione shiftato di prima forma:} identico alla prima forma ma la base
invece che 0, parte da $k\leq n$
\begin{itemize}
    \item base induttiva: P(k) è vera
    \item passo induttivo: $\forall n\geq k$, P(n) è vera $\Rightarrow$ P(n+1)
\end{itemize}

\section{third week}
exercises

\newpage
\section{forth week}

\textbf{Insiemi finiti:} Indicando con I$_{n}$ un insieme che va da 0 a n, diremo che l'insieme X è finito
se essite $n\in\mathbb{N}$ t.c. $I_{n}\sim X$. Se non esiste lo definiremo insieme infinito.

\textbf{Teorema di lemma dei cassetti:} Siano X e Y due insiemi rispettivamente $X\sim I_{n}$ e $Y\sim I_{m}$
con $n<m$ allora la funzione $f(x):\ Y\rightarrow X$ non è iniettiva

\textbf{Cardinalità:} Sia X un insieme finito. Definiamo cardinalità n t.c. $I_{n}$ sia equipotente
a X. Definiamo $I_{n}$ come insieme cardinalità associato a X

\textbf{Proposizione:} Sia $A$ insieme finito e $B\subseteq A$, allora $|B|\leq |A|$

\textbf{Osservazione:} Qualsiasi $f(x): \mathbb{N}\rightarrow\mathbb{N}/\{0\} $ è bigiettiva

\textbf{Minimo:} Sia $A$ un insieme e $z\in A$. Se $\forall x \in A, z\leq x$, allora definiremo $z$ come 
\textbf{minimo} di $A$: 
\begin{center}
    $z = min(A)$
\end{center}

\textbf{Buon ordinamento:} Un ordinamento totale è definito \textbf{ben ordinato} se ogni sottoinsieme di 
$(Z,\leq)$ ammette un minimo

\textbf{Assioma di buon ordinamento:} L'ordinamento $(\mathbb{N},\leq)$ è ben ordinato e l'ordinamento $\leq$
è \textbf{usuale} su $\mathbb{N}$ (cioé se $\exists k\ t.c.\ n+k=m$ allora $n\leq m$)

\textbf{Principio di induzione (2$^{a} forma$):} prendiamo una famiglia di preposizioni $P(n)$ e supponiamo
rispetti le 2 condizioni:
\begin{itemize}
    \item la base induttiva: $P(0)$ è vera
    \item il passo induttivo: $\forall n\in\mathbb{N}, \forall k \in\mathbb{N}\ t.c.\ 0\leq k \leq n$,
    $P(k)$ è vera (ipotesi induttiva), allora P(n)
\end{itemize}
Se rispetta questa condizioni allora implica $\forall n\in\mathbb{N},\ P(n)$

\newpage
\textbf{Divisione euclidea:} Siano $n,m\in\mathbb{Z} t.c. m\neq 0 \exists ! q,r\in\mathbb{Z} t.c:$
\begin{center}
    $n=mq+r$\\$0\leq n <|m|$
\end{center}
(si definiscono $q$ quoziente e $r$ resto della divisione di $n$ per $m$)

\newpage
\section{fifth week}

\textbf{Rappresentabilità:} Sia $b\in\mathbb{N}$, diremo che $n\in\mathbb{N}$ è rappresentabile in base
$b$ se esistono $k\in\mathbb{N}$ e $\varepsilon{0},\varepsilon{1},...,\varepsilon{k}\in I_{b}\ t.c:$
\begin{center}
    \ \ \ \ \ \ \ \ \ \ \ \ \ \ $n = \sum_{i=0}^{k}\ \varepsilon_{i}b^{i}$\ \ \ \ \ \ \ \ con $I_{b}=\{0,1,...,b-1\}$
\end{center} 

\textbf{Teorema della rappresentazione dei naturali in base arbitraria:} Sia $b\in\mathbb{N}, b\geq 2$, allora
$\forall n\in\mathbb{N},\ n$ è rappresentabile in base $b$ in maniera univoca

\textbf{Divisibilità:} Dati $n,m\in\mathbb{Z}$ si dice che $n$ è \textbf{divisore} di $m$ (o $m$ è multiplo
di $n$) se $\exists k\in\mathbb{Z}\ t.c.\ m=nk$ e scriveremo $n|m$

\textbf{Proprità della divisibilità:}
\begin{itemize}
    \item se $n|m$ e $m|q$ allora $n|q$
    \item se $n|m$ e $m|n$ allora $n=\pm m$
\end{itemize}

\textbf{Massimo Comune Divisore:} Dati $m,n\in\mathbb{Z}$ si dice $d\in\mathbb{Z},\ d>0$ massimo comune divisore
se:
\begin{itemize}
    \item $d|n\ e\ d|m$
    \item $\exists c\in\mathbb{Z}\ t.c.\ c|n\ c|m\ c|d$
\end{itemize} 

proposizione: se $d$ e $d^{I}$ sono mcd tra $m$ e $n$ allora $d=d^{I}$

\textbf{Teorema:} dati $n,m\in\mathbb{Z} \neq 0$, esiste mcd unico indicato con $(n,m)$

\textbf{Lemma utile:} dati $n,m,c\in\mathbb{Z} \neq 0$ e $c|n\ c|m$, allora $\forall x,y\in\mathbb{Z}$ vale:
\begin{center}
    $c|xn+ym$
\end{center}

Corollario: Siano $n,m\in\mathbb{Z} \neq 0$ se sia $d:=(n,m)$ allora esistono $x,y\in\mathbb{Z}\ t.c:$
\begin{center}
    $d=xn+ym$
\end{center}

\textbf{Numeri coprimi:} dati $n,m\in\mathbb{Z}$, si dicono coprimi fra di loro se $(n,m)=1$

proposizione: sia $d=n,m$ allora $(\frac{n}{d},\frac{m}{d})=1$

\textbf{Algoritmo di Euclide:}
\begin{center}
    \begin{tabular}{ c c c }
        
        $n=q_{1}m+r_{1}$ & & $r_{1}=n-q_{1}m$\\
        $m=q_{2}r_{1}+r_{2}$ & & $r_{2}=m-q_{2}r_{1}$\\
        $r_{1}=q_{3}r_{2}+r_{3}$ & &$r_{3}=r_{1}-q_{3}r_{2}$\\
        $\cdot$ & &$\cdot$\\
        $\cdot$ & $\Rightarrow$ & $\cdot$\\
        $\cdot$ & & $\cdot$\\
        $r_{k-3}=q_{k-1}r_{k-2}+r_{k-1}$ & &$r_{k-1}=r_{k-3}-q_{k-1}r_{k-2}$\\
        $r_{k-2}=q_{k}r_{k-1}+r_{k}$ & & $r_{k}=r_{k-2}-q_{k}r_{k-1}$\\
        $r_{k-1}=q_{k+1}r_{k}+0$ & & $= xn+ym$
        
    \end{tabular}
\end{center}

Es:
\begin{center}
    (48,28)

    \begin{tabular}{ c c c }
        $48=28\cdot 1 + 20$ & &\\
        $28=20\cdot 1 + 8$ && $4=20-2\cdot 8$\\
        $20=8\cdot 2 + 4$ &$\Rightarrow$& $8=28-20\cdot 1$\\
        $8=\underline{4}\cdot 2 + 0$ &&$20 = 48-28\cdot 1$\\
        $MDC = 4$&&\\
    \end{tabular}

    $\Downarrow$\\
    $4=20-2(28-20\cdot 1)=3\cdot 20 - 2\cdot 28$\\
    $4=3(48-28\cdot 1)-2\cdot 28$\\
    $=\underline{3\cdot 48 - 5\cdot 28}$

\end{center}

\newpage
\section{Sixth week}

\textbf{Proprietà dei coprimi:} Siano $n,m,q\in\mathbb{Z}$ e $n$ o $m\neq 0$ e $(n,m)=1$:
\begin{itemize}
    \item Se $n|mq$ allora $n|q$
    \item Se $n|q$ e $m|q$ allora $nm|q$
\end{itemize} 

\textbf{Numeri primi:} $p\in\mathbb{Z}$ si dice \textbf{primo} se $p\geq 2$ e i suoi divisori sono quelli banali 
($\pm 1|p,\ \pm p|p$). $p$ è primo se $\forall n,m$ e $p|nm$ allora $p|n\bigvee p|m$

\textbf{Minimo Comune Multiplo:} dati $n,m\in\mathbb{Z}$ si dice $M$ minimo comune multiplo di $n$ e $m$ se:
\begin{itemize}
    \item $n|M$ e $m|M$
    \item $\exists c\ t.c.\ n|c, m|c, M|c$
\end{itemize}

\textbf{Unicità mcm:} dati $n,m\in\mathbb{Z}$ e $M,M^{1}$ sono mcm di $n$ e $m$, allora $M=M^{1}$

\textbf{Denotazione mcm:} mcm di $n$ e  $m$ si scrive $[n,m]$

\textbf{Teorema d'esistenza:} siano $n,m\in\mathbb{Z}$ allora $\exists [n,m]$ e se $n\bigvee m\neq 0$ allora:
\begin{center}
    $[n,m] = \frac{nm}{(n,m)}$
\end{center}

\textbf{Teorema fondamentale dell'aritmetica:} $\forall n\in\mathbb{N},n\geq 2$, $n$ è uguale a un prodotto di numeri
primi, anche ripetuti:
\begin{center}
    $n=p_{1}\cdot p_{2}\cdot p_{3}\cdot ... \cdot p_{k}$
\end{center}
La fattorizzazione di questo prodotto è univoca

Corollario: i numeri primi sono infiniti

\textbf{Congruenza:} dati $a,b\in\mathbb{Z}$ diremo che $a$ è  congruo a $b$ modulo $n$ ($a\equiv b\ mod\ n$) se
\begin{center}
    $n|a-b$
\end{center} 

\textbf{Proprietà congruenza:} 
\begin{itemize}
    \item riflessiva: $a\equiv a\ mod\ n\ \ \ \forall a,n\in\mathbb{Z}$
    \item simmetrica: $a\equiv b\ mod\ n$ allora $b\equiv a\ mod\ n\ \ \ \forall a,b,n\in\mathbb{Z}$
    \item transitiva: $a\equiv b\ mod\ n$ e $b\equiv c\ mod\ n$ allora $a\equiv c\ mod\ n\ \ \ 
    \forall a,b,c,n\in\mathbb{Z}$
\end{itemize}

\textbf{equivalenza:} una relazione $\mathcal{R}$ binaria su l'insieme $X$ si dice relazione d'equivalenza su $X$ se:
\begin{itemize}
    \item è riflessiva: $\forall x\in X,\ x\mathcal{R}x$
    \item è simmetrica: $\forall x,y\in X,\ x\mathcal{R}y$ allora $y\mathcal{R}x$
    \item è transitiva: $\forall x,y,z\in X,\ x\mathcal{R}y$ e $y\mathcal{R}z$ allora $x\mathcal{R}z$
\end{itemize}

\newpage
\section{seventh week}

\textbf{Classi d'equivalenza:} sia $X$, $x\in X$ e $\sim$ una relazione d'equivalenza su $X$. Chiameremo classe
d'equivalenza di $x$ in $X$ rispetto a $\sim$ il sottoinsieme di $X$ i quali elementi $y$ sono equivalenti a $x$:
\begin{center}
    $[x]_{\sim}=\{y\in x| y\sim x\}$
\end{center}

\textbf{Insieme quoziente:} chiameremo insieme quoziente di $X$ modulo $\sim$ l'insieme delle classi d'equivalenza
contenute in $X$:
\begin{center}
    $X/\sim =\{ y\in\ x| y\sim x\}$
\end{center}

\textbf{Proprietà classi d'equivalenza:}
\begin{itemize}
    \item $\forall x\in X,\ x\in [x]$
    \item $\forall x,y\in X,\ [x]=[y] \Leftrightarrow x\sim y$
    \item $\forall x,y\in X,\ [x]\cap [y]\neq 0\Rightarrow [x]=[y]$
\end{itemize}

\textbf{Classi di congruenza:} Dati $a,n\in\mathbb{Z}$ definiamo la classe di congruenza di $a$ modulo
$n$ l'insieme delle $x$ congruenti ad $a\ mod\ n$:
\begin{center}
    $[a]_{n}=\{x\in\mathbb{Z}|x\equiv a\ mod\ n\}$
\end{center}
Indicheremo l'insieme quoziente $\mathbb{Z}\ mod\ \sim_{n}$ come $\mathbb{Z}/_{n}\mathbb{Z}$ e ha 
come elementi le classi di congruenza $[a]_{n}$ che appartengono alle partizioni di $\mathbb{Z}$
($2^{\mathbb{Z}}$), quindi:
\begin{center}
    $[a]_{n}=\{a+kn|k\in\mathbb{Z}\}$
\end{center}
Es:\\ $\mathbb{Z}/_{3}\mathbb{Z}=\{[0]_{3},[1]_{3},[2]_{3}\}$

\newpage

\textbf{Prop:} Sia $a\in\mathbb{Z}$ e sia $r$ il resto di $\frac{a}{n}$, allora $a\equiv r\ (mod\ n)$, oppure:
\begin{center}
    $[a]_{n}=[r]_{n}$
\end{center}

\textbf{Criterio di divisibilità:} dati $a,n\in\mathbb{Z}$ con $n\neq 0$, diremo che $a$ è multiplo di $n$ se:
\begin{center}
    $[a]_{n}=[0]_{n}$
\end{center}

\textbf{Notazione:} dato $a\in\mathbb{Z}$ e $x\in [a]_{n}$ ($[a]_{n}=[x]_{n}$), diremo che $x$ è \textbf{rappresentante
della classe} $[a]_{n}$. Se $x$ è di tipo resto, allora $x$ è \textbf{rappresentante canonico}

gli elementi di $\mathbb{Z}/_{n}\mathbb{Z}$ si chiamano \textbf{classi di resto} modulo $n$

\textbf{Struttura algebrica:} esistono due operazioni di somma e moltiplicazione tra insiemi quozienti:
\begin{itemize}
    \item Somma: $[a]_{n}+[b]_{n}=[a+b]_{n}$
    \item Moltiplicazione: $[a]_{n}\cdot [b]_{n}=[a\cdot b]_{n}$
\end{itemize}

\textbf{Prop:} dati $a,a^{1},b,b^{1}\in\mathbb{Z}$ tc $[a]_{n}=[a^{I}]_{n}$ e $[b]_{n}=[b^{I}]_{n}$ allora:
\begin{itemize}
    \item Somma: $[a+b]_{n}=[a^{I}+b^{I}]_{n}$
    \item Moltiplicazione: $[a\cdot b]_{n}=[a^{I}\cdot b^{I}]_{n}$
\end{itemize}

Oss: Sia $a\in\mathbb{Z},\ m\in\mathbb{N},\ m>0$. Allora:
\begin{center}
    $[a]_{n}^{m}=[a_{1}]_{n}\cdot[a_{2}]_{n}\cdot ... \cdot[a_{m}]_{n}\cdot = [a^{m}]_{n}$
\end{center}

\newpage
\section{eight week}
\textbf{Teorema cinese del resto:} Siano $n,m>0$ e siano $a,b\in\mathbb{Z}$. Consideriamo il seguente
sistema di congruenze:
\begin{center}
    \begin{tabular}{ c c c }
        $\begin{cases}
            x\in\mathbb{Z}\\
            x\equiv a(mod\ n)\\
            x\equiv b(mod\ m)
        \end{cases}$
        & o &
        $\begin{cases}
            x\in\mathbb{Z}\\
            [x]_{n}=[a]_{n}\\
            [x]_{m}=[b]_{m}
        \end{cases}$
    \end{tabular}
\end{center}

Sia $S$ l'insieme delle soluzioni dei precedenti Sistemi
\begin{center}
    $S=\langle x\in\mathbb{Z}|x\equiv a(mod\ n)\ e\ x\equiv b(mod\ m) \rangle$
\end{center}
Il precedente sistema è \textbf{compatile} (ammette soluzioni) se e soltanto se:
\begin{center}
    $(n,m)|a-b$
\end{center}
Se $S\neq\varnothing$ e $c\in S$, allora $S=[c]_{[n,m]}\in\mathbb{Z}=\langle c+k_{[n,m]}\in\mathbb{Z}
|k\in\mathbb{Z}\rangle$

Es:

$
\begin{cases}
 x\equiv 9(mod\ 162)\\
 x\equiv -9(mod\ 114)
\end{cases}
$

1 - Compatibilità
\begin{center}
    \begin{tabular}{ c c c }  
        $(162,114) = 6$ & $\Rightarrow$ & $(162,114)| 9-(-9) = 6|18 = 3$\\
        && $\Rightarrow 9-(-9) = 3(162,114)_{(1)}$
    \end{tabular}
\end{center}

2 - Calcolo di una soluzione\\
Algoritmo di Euclide:
\begin{center}
    \begin{tabular}{c|c c c}
        $162=114+48$ & $48=162-114$ && $=18-(48-2\cdot 18)= 3\cdot 18-48$\\
        $114=2\cdot 48+18$ & $18=114-2\cdot 48$ && $=3(114-2\cdot 48)-48 = 3\cdot 114-7\cdot $$48$\\
        $48=2\cdot 18+12$ & $12=48-2\cdot 18$ & $\Rightarrow$ & $=3\cdot 114-7(162-114)=10\cdot 114-7\cdot 162$\\
        $18=12+6$ & $6=18-12$ && $6 = 10\cdot 114-7\cdot 162$\\
        $12=2\cdot 6+0$&&& $(162,114)=10\cdot 114-7\cdot 162_{(2)}$\\
    \end{tabular}
\end{center}
\newpage
Da (1) e (2) segue che 
\begin{center}
    $9-(-9)=3(162,114)=3(10\cdot 114-7\cdot 162)$\\
    $9-(-9)= 30\cdot 114-21\cdot 162_{(3)}$

    $9+21\cdot 162=-9+30\cdot 114 \Rightarrow 3411$
\end{center}
$c=3411$ è una soluzione del sistema

3 - Calcolo di S\\
Teorema cinese del resto:
\begin{center}
    $S=[c]_{[162,114]}=[3411]_{[162,114]}$

    \begin{tabular}{c c c}
        $[162,114]=\frac{162\cdot 114}{(162,114)}=3078$ &$\Rightarrow$ & $S=[3411]_{[3078]}=[333]_{[3078]}$
    \end{tabular}
    
$\Rightarrow S=\langle 333+3078k\in\mathbb{Z}|k\in\mathbb{Z}\rangle$
\end{center}
Bonus:\\
Esiste soluzione di $S$ divisibile da 17?

metodo 1
\begin{center}
    $\begin{cases}
        x\equiv 333(mod\ 3078)\\
        x\equiv 0(mod\ 17)
    \end{cases}$

    $(3078,17)|333-0$

    $1|333$
\end{center}
è divisibile quindi accetta soluzione

metodo 2
\begin{center}
    $[333+3078k]_{17}=[333]_{17}+[3078]_{17}[k]_{17}$\\
    $[10]_{17}+[1]_{17}[k]_[17]=[10+k]_{17}$\\
    $\Rightarrow k=7$
\end{center}

\newpage
\textbf{Elementi invertibili modulo n:}
Siano $a,n\in\mathbb{Z}$ con $n>0$. Diremo che $a$ è invertibile modulo $n$ o equivalentemente che $[a]_{n}$ è
invertibile in $\mathbb{Z}/_{n}\mathbb{Z}$ se esiste $x\in\mathbb{Z}$ tc:
\begin{center}
    $ax\equiv 1(mod\ n) \Leftrightarrow [a]_{n}[x]_{n}=[1]_{n}$
\end{center}
In questo caso diremo che $x$ è un'inversa di $a(mod\ n)$ e $[x]_{n}$ è una classe inversa di $[a]_{n}$ in 
$\mathbb[Z]/_{n}\mathbb{Z}$

\textbf{Lemma:} Supponiamo che $a$ sia invertibile modulo $n$, ovver $[a]_{n}$ sia invertibile in
$\mathbb[Z]/_{n}\mathbb{Z}$. Allora esiste un unico $[x]_{n}\in\mathbb[Z]/_{n}\mathbb{Z}$ tale che:
\begin{center}
    $[a]_{n}[x]_{n}=[x]_{n}[a]_{n}=[1]_{n}$
\end{center}
Equivalentemente $[x]_{n}$ è l'unica classe inversa di $[a]_{n}$ in $\mathbb[Z]/_{n}\mathbb{Z}$.
Tale classe $[x]_{n}$ viene detta inversa e viene indicata con il simbolo $[a]_{n}^{-1}$

\textbf{Prop:} $a\in\mathbb{Z}$ è invertibile $mod\ n\Leftrightarrow(a,n)=1$, in questo caso esiste $x,y\in\mathbb{Z}$
tali che:
\begin{center}
    $xa+yn=1$\\
    $(Algoritmo\ di\ euclide)$
\end{center}
Allora
\begin{center}
    $[a]_{n}^{-1}=[x]_{n}$
\end{center}

Es:\\
$11\ inv(mod\ 30)$

$(11,30) = 1 \Rightarrow \exists[11]_{30}^{-1}$

alg. euclide:\\
$1= 11\cdot 11 + (-4)30$

$[1]_{30}=[(11)(11)+(-4)(30)]=[11]_{30}[11]_{30}+[-4]_{30}[0]_{30}=\underline{[11]_{30}}[11]_{30} \Rightarrow 
[11]_{30}^{-1}=[11]_{30}$

\textbf{Def:} Dato $n\in\mathbb{Z}, n>0$, indichiamo con $(\mathbb{Z}/_{n}\mathbb{Z})$* il sottoinsieme di 
\marginpar{invertibili, cioé mcd è uguale a 1}
$\mathbb{Z}/_{n}\mathbb{Z}$ formato da tutti gli interi modulo $n$ invertibili

\textbf{Prop:} Sia $p$ numero primo, allora vale:
\begin{center}
    $(\mathbb{Z}/_{n}\mathbb{Z})$*$=\{[1]_{p},[2]_{p},...,[p-1]_{p}\}=\mathbb{Z}/_{p}\mathbb{Z}\backslash \{[0]_{p}\}$
\end{center}
\end{document}