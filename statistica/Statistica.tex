\documentclass[12pt, letterpaper]{article}
\usepackage[utf8]{inputenc}
\usepackage{fixltx2e}
\usepackage{amsmath}
\title{Statistica}
\date{}
\author{}

\def\SPSB#1#2{\rlap{\textsuperscript{#1}}\SB{#2}}
\def\SP#1{\textsuperscript{#1}}
\def\SB#1{\textsubscript{#1}}

\begin{document}

\section{Introduzione}

probabilità $\rightarrow$ misurare l'incertezza

statistica:
\begin{itemize}
    \item descrittiva
    \item differenziale $\rightarrow$  \underline{campione casuale} per \underline{stimare} un esito
\end{itemize}

\begin{center}
    \rule{100pt}{0.5pt}
\end{center}

probabilità:  
\begin{center}
    $\frac{casi favorevoli}{casi totali}$
\textbf{\underline{SE}} equiprobabili
\end{center}

per contare i casi ci si appoggia alla \underline{combinatoria}

\begin{center}
    \rule{100pt}{0.5pt}
\end{center}

partizione: separazione di A in sottoinsiemi senza elementi comuni

\begin{center}
    \rule{100pt}{0.5pt}
\end{center}

\textbf{NB:} 
\begin{itemize}
    \item $\wedge$ - and $\rightarrow$ A$\cap$B = \{x$|$x$\in$A$\wedge$x$\in$B\}
    \item $\vee$ - or $\rightarrow$ A$\cup$B = \{x$|$x$\in$A$\vee$x$\in$B\}
\end{itemize} 

\begin{center}
    \rule{100pt}{0.5pt}
\end{center}

\textbf{Principi della combinatoria:}
\begin{enumerate}
    \item A insieme, \{E\textsubscript{i}\}\SPSB{n}{i=1} partizione di A $\rightarrow$ \#A = $\sum_{i = 1}^{n}$ \#E\textsubscript{i} 
    \begin{itemize}
        \item A,B insiemi, AxB è l'insieme di coppie ordinate (a,b)
    \end{itemize}  
    \item \#(AxB) = \#A$\cdot$\#B $\rightarrow$ \{A\textsubscript{i}\}\SPSB{n}{i = 1} = $\bigotimes$\SPSB{n}{i=1} A\SB{i}
    \item A,B, \#(A$\cup$B) = \underline{\#A + \#B - \#(A{$\cap$}B)} (non perfetto) $\rightarrow$
    \begin{center}
        \item \# $\cup$\SPSB{n}{i= 1} A\SB{i} = $\sum_{i = 1}^{n}$ \#A\SB{i} - $\sum_{i<j}^{}$\#(A\SB{i}$\cap$A\SB{j}) +
        $\sum_{i<j<k}^{}$\#(A\SB{i}$\cap$A\SB{j}$\cap$A\SB{k}) + \dots
        
        $\downarrow$

        \textbf{+(-1)\SP{n+1} \#$\cap$\SPSB{n}{i=1}A\SB{i}}
    \end{center}
\end{enumerate}

\section{Permutazioni e anagrammi}

Fattoriale $\rightarrow$ x! = 9! = 9$\cdot$8$\cdot$7$\cdot$6$\cdot$$\dots$$\cdot$2$\cdot$1

\textbf{NB:} 0! = 1

\begin{itemize}
    \item "prendiamo" ha 9! anagrammi
    \item "anagramma" ha tre ripetizioni di a e due ripetizioni di m, quindi per calcolare i casi unici:   
\end{itemize}
\begin{center}
    \Large $\frac{9!}{3!{\cdot}2!}$

    $\downarrow$
 \end{center} 
per calcolare la probabilitàdegli elementi n, ma mi interessano solo k elementi allora:
\begin{center}
    \Large $\frac{n!}{(n-k)!}$
\end{center}
se non sono interessato all'ordine, allora:
\begin{center}
    \Large $\frac{n!}{(n-k)!k!}$ $\Rightarrow$ $\binom{n}{k}$ 
\end{center}
chiamato anche \textbf{coefficente binominiale}
\\
\textbf{Proprietà:}
\begin{itemize}
    \item $\binom{n}{k}$ = $\binom{n}{n-k}$
    \item $\binom{n}{0}$ = $\binom{n}{n}$ = 1
    \item $\sum_{k=0}^{n}$ $\binom{n}{k}$  = 2\SP{n}
    \item $\binom{n}{k}$ + $\binom{n}{k+1}$ = $\binom{n+1}{k+1}$  
\end{itemize}

\section{Esperimenti aliatori}

Un esperimento si definisce \textbf{aliatorio} o casuale se con i dati iniziali il risultato è incerto.
\\
I risultati a 2 a2 incompatibili di un esperimento aliatorio sono chiamati \textbf{esiti}.
\\
$\Omega$ denota lo \textbf{spazio degli esiti}.
\\
Un \textbf{evento} è un osservabile di un esperimento aliatorio.
\\
Una parte di $\Omega$ può essere considerata come famiglia:
\begin{center}
    $\mathcal{F}$ $\subseteq$ P($\Omega$)
\end{center}
Questa è definitò come \textbf{algebra} se:
\begin{itemize}
    \item $\Omega$$\in$$\mathcal{F}$
    \item se A$\in$$\mathcal{F}$ allora A\SP{c}$\in$$\mathcal{F}$
    \item se A,B$\in$$\mathcal{F}$, allora A$\cup$B$\in$$\mathcal{F}$
    \subitem potremmo scrivere anche \{A\SB{i}\}\SPSB{n}{i=1}$\subseteq$$\mathcal{F}$ allora $\cup$\SPSB{n}{i=1}A\SB{i}$\in$$\mathcal{F}$
\end{itemize}

\large\textbf{Proprietà}
\begin{itemize}
    \item $\emptyset$$\in$$\mathcal{F}$
    \item se A,B$\in$$\mathcal{F}$ allora A$\cap$B$\in$$\mathcal{F}$
    \item se \{A\SB{i}\}\SPSB{n}{i=1}$\subseteq$$\mathcal{F}$ allora $\cap$\SPSB{n}{i=1}A\SB{i}$\in$$\mathcal{F}$
\end{itemize}
\end{document}