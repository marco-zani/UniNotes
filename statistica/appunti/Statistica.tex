\documentclass[11pt, letterpaper]{article}
\usepackage[utf8]{inputenc}
\usepackage{fixltx2e}
\usepackage{amsmath}
\title{Statistica}
\date{}
\author{}

\setlength{\parindent}{0em}
\setlength{\parskip}{1em}

\def\SPSB#1#2{\rlap{\textsuperscript{#1}}\SB{#2}}
\def\SP#1{\textsuperscript{#1}}
\def\SB#1{\textsubscript{#1}}

\def\separator{\begin{center}    \rule{100pt}{0.5pt}\end{center}}

\begin{document}

\section{Introduzione}

probabilità $\rightarrow$ misurare l'incertezza\\
statistica:
\begin{itemize}
    \item descrittiva
    \item differenziale $\rightarrow$  \underline{campione casuale} per \underline{stimare} un esito
\end{itemize}

\separator

probabilità:  
\begin{center}
    $\frac{casi favorevoli}{casi totali}$
\textbf{\underline{SE}} equiprobabili
\end{center}

per contare i casi ci si appoggia alla \underline{combinatoria}

\separator

partizione: separazione di A in sottoinsiemi senza elementi comuni

\separator

\textbf{NB:} 
\begin{itemize}
    \item $\wedge$ - and $\rightarrow$ A$\cap$B = \{x$|$x$\in$A$\wedge$x$\in$B\}
    \item $\vee$ - or $\rightarrow$ A$\cup$B = \{x$|$x$\in$A$\vee$x$\in$B\}
\end{itemize} 


\textbf{Principi della combinatoria:}
\begin{enumerate}
    \item A insieme, \{E\textsubscript{i}\}\SPSB{n}{i=1} partizione di A $\rightarrow$ \#A = $\sum_{i = 1}^{n}$ 
    \#E\textsubscript{i} 
    \begin{itemize}
        \item A,B insiemi, AxB è l'insieme di coppie ordinate (a,b)
    \end{itemize}  
    \item \#(AxB) = \#A$\cdot$\#B $\rightarrow$ \{A\textsubscript{i}\}\SPSB{n}{i = 1} = $\bigotimes$\SPSB{n}{i=1} A\SB{i}
    \item A,B, \#(A$\cup$B) = \underline{\#A + \#B - \#(A{$\cap$}B)} (non perfetto) $\rightarrow$
    \begin{center}
        \item \# $\cup$\SPSB{n}{i= 1} A\SB{i} = $\sum_{i = 1}^{n}$ \#A\SB{i} - $\sum_{i<j}^{}$\#(A\SB{i}$\cap$A\SB{j}) +
        $\sum_{i<j<k}^{}$\#(A\SB{i}$\cap$A\SB{j}$\cap$A\SB{k}) + \dots
        
        $\downarrow$

        \textbf{+(-1)\SP{n+1} \#$\cap$\SPSB{n}{i=1}A\SB{i}}
    \end{center}
\end{enumerate}




\section{Permutazioni e anagrammi}

Fattoriale $\rightarrow$ x! = 9! = 9$\cdot$8$\cdot$7$\cdot$6$\cdot\dots\cdot$2$\cdot$1

\textbf{NB:} 0! = 1

\begin{itemize}
    \item "prendiamo" ha 9! anagrammi
    \item "anagramma" ha tre ripetizioni di a e due ripetizioni di m, quindi per calcolare i casi unici:   
\end{itemize}
\begin{center}
    \Large $\frac{9!}{3!{\cdot}2!}$ \normalsize

    $\downarrow$
 \end{center} 
per calcolare la probabilitàdegli elementi n, ma mi interessano solo k elementi allora:
\begin{center}
    \Large $\frac{n!}{(n-k)!}$ \normalsize
\end{center}
se non sono interessato all'ordine, allora:
\begin{center}
    \Large $\frac{n!}{(n-k)!k!}$ $\Rightarrow$ $\binom{n}{k}$ \normalsize
\end{center}
chiamato anche \textbf{coefficente binominiale}

\separator

\textbf{Proprietà:}
\begin{itemize}
    \item $\binom{n}{k}$ = $\binom{n}{n-k}$
    \item $\binom{n}{0}$ = $\binom{n}{n}$ = 1
    \item $\sum_{k=0}^{n}$ $\binom{n}{k}$  = 2\SP{n}
    \item $\binom{n}{k}$ + $\binom{n}{k+1}$ = $\binom{n+1}{k+1}$  
\end{itemize}




\section{Esperimenti aliatori}

Un esperimento si definisce \textbf{aliatorio} o casuale se con i dati iniziali il risultato è incerto.
I risultati a 2 a2 incompatibili di un esperimento aliatorio sono chiamati \textbf{esiti}.
$\Omega$ denota lo \textbf{spazio degli esiti}.
Un \textbf{evento} è un osservabile di un esperimento aliatorio.

Una parte di $\Omega$ può essere considerata come famiglia:
\begin{center}
    $\mathcal{F}$ $\subseteq$ P($\Omega$)
\end{center}
Questa è definita come \textbf{algebra} se:
\begin{itemize}
    \item $\Omega\in\mathcal{F}$
    \item se A$\in\mathcal{F}$ allora A\SP{c}$\in\mathcal{F}$
    \item se A,B$\in\mathcal{F}$, allora A$\cup$B$\in\mathcal{F}$
    \subitem potremmo scrivere anche \{A\SB{i}\}\SPSB{n}{i=1}$\subseteq\mathcal{F}$ allora 
    $\cup$\SPSB{n}{i=1}A\SB{i}$\in\mathcal{F}$
\end{itemize}

\separator

\large{\textbf{Proprietà}}\normalsize
\begin{itemize}
    \item $\emptyset\in\mathcal{F}$
    \item se A,B$\in\mathcal{F}$ allora A$\cap$B$\in\mathcal{F}$
    \item se \{A\SB{i}\}\SPSB{n}{i=1}$\subseteq\mathcal{F}$ allora $\cap$\SPSB{n}{i=1}A\SB{i}$\in\mathcal{F}$
    \item se A,B $\in\mathcal{F}$, allora A$\cdot$B $\in\mathcal{F}$
    \item se A,B $\in\mathcal{F}$, allora A$\triangle$B $\in\mathcal{F}$
\end{itemize}

\separator
 
$\mathcal{F}\subseteq$P($\Omega)$ è una \textbf{tribù} se:
\marginpar{\small NB: generalmente una tribù è un'algebra se hanno elementi finiti}
\begin{itemize}
    \item $\Omega\in\mathcal{F}$ 
    \item A$\in\mathcal{F}\Rightarrow$ A\SP{C}$\in\mathcal{F}$
    \item per ogni famiglia \underline{numerabile} \{A\SB{i}\}\SPSB{+$\infty$}{i=1}$\subseteq$P($\Omega$),\\ allora 
    $\cup$\SPSB{+$\infty$}{i=1}A\SB{i}$\in\mathcal{F}$
\end{itemize}

$\mathcal{F}$ tribù su $\Omega$. Ogni E$\in\mathcal{F}$ (E è sottoinsieme di $\Omega$) si dice \textbf{Evento}.
I singoletti si chiamano \textbf{eventi elementari}. E si verifica se il risultato dell'esperimento appartiene ad E
$\mathcal{F}$ tribù su $\Omega$ ($\Omega$,$\mathcal{F}$)\\

Dati $\Omega$, $\mathcal{F}$ tribù su $\Omega$ ($\Omega$,$\mathcal{F}$) si chiama \textbf{spazio probabilizzabile}.

($\Omega$,$\mathcal{F}$), una funzione P:$\mathcal{F}\rightarrow\mathcal{R}$ si dice
\textbf{funzione di probabilità} se:
\begin{itemize}
    \item per ogni evento E P(E)$\geq$0
    \item P($\Omega$)=1 
    \item data una famiglia numerabile \{E\SB{i}\}\SPSB{+$\infty$}{i=1} di eventi a 2 a 2 disgiunti:
    \begin{center}
        P($\cup$\SPSB{$\infty$}{i=1}E\SB{i}) = $\sum_{i=1}^{\infty}$P(E\SB{i}) (additività)
    \end{center}
\end{itemize}


\separator

\large{\textbf{Proprietà delle probabilità}}\normalsize
\begin{itemize}
    \item P($\emptyset$) = 0
    \item E$\in\mathcal{F}$ allora P(E\SP{c}) = 1-P(E)
    \item E,F$\in\mathcal{F}$, E$\subseteq$F $\Rightarrow$ P(E)$\leq$P(F)
    \subitem E$\in\mathcal{F}$ P(E)$\leq$1
    \item E,F$\in\mathcal{F}$ P(E$\cup$F)=P(E)+P(F)-P(E$\cap$F)
    \subitem P(E$\cup$F)$\leq$P(E)+P(F)
    \subitem (E\SB{i})\SPSB{n}{i=1}<$\mathcal{F}$, P($\cup$\SPSB{n}{i=1}E\SB{i})=$\sum_{k\in\mathcal{P}(\{1-n\})}$(-1)\SP{\#k+1}
    P($\cap$\SB{j$\in$k}E\SB{j})
    \subitem (E\SB{i})\SPSB{+$\infty$}{i=1}$\subset\mathcal{F}$, P($\cup$\SPSB{$\infty$}{i=1}E\SB{i})$\leq\sum_{i=1}^{\infty}$ 
    P(E\SB{i})
    \item (disuguaglianza di bonferrow) \\
    $\sum_{i=1}^{+\infty}$ P(E\SB{i}) - $\sum_{i<j}$ P(E\SB{i}$\cap$E\SB{j}) $\leq$ P($\cup$\SPSB{+$\infty$}{i=1}E\SB{i}) $\leq$ $\sum_{i=1}^{\infty}$ P(E\SB{i})
\end{itemize}




\section{Probabilità condizionata}

($\Omega$, $\mathcal{F}$, P), E,F$\in\mathcal{F}$ con P(F)$\neq$0, allora la probabilità di E condizionale a F è:

\begin{center}
    P(E$|$F) = $\frac{P(E\cap F)}{P(F)}$
\end{center}


Dato ($\Omega$, $\mathcal{F}$, P) e due sotto tribù $\mathcal{F}\SB{1},\mathcal{F}\SB{2}$ di $\mathcal{F}$
allora $\mathcal{F}\SB{1}e\mathcal{F}\SB{2}$ sono indipendenti se se ogni elemento di $\mathcal{F}\SB{1}$ è indipendente
da ogni elemento di $\mathcal{F}\SB{2}$
\begin{center}
P(E\SB{1}$\cap$E\SB{2} $\vert\mathcal{F}$) = P(E\SB{1}$\vert\mathcal{F}$) $\cdot$ P(E\SB{2}$\vert\mathcal{F}$)    
\end{center}

\clearpage
\section{funzione di probabilità ($\Omega,\mathcal{F},P$)}

\begin{enumerate}
    \item $\Omega$ finito o numerabile
    \subitem $\Omega$ è dato 
    \subitem $\mathcal{F}=\mathcal{P}(\Omega)$
    \subitem P: assegnamo ad ogni singoletto ($\omega\in\Omega$) un probabilità tale che:
    \subsubitem P($\omega$)$\geq$0
    \subsubitem $\sum$P($\omega$)=1
    \subitem A questo punto $\forall$ E $\in\mathcal{F}$ P(E):=$\sum_{\omega\in E}$P($\omega$)
    \item Spazi prodotto
    \subitem considerando più ripetizioni di un esperimento o l'unione di più esperimenti:
    data una famiglia di sottoinsiemi di $\Omega$ dette $\mathcal{A}$. La tribù di $\mathcal{F}$\SB{$\mathcal{A}$}
    generata da $\mathcal{A}$ come la più piccola tribù contenente $\mathcal{A}$
    \begin{center}
        $\mathcal{F}$\SB{$\mathcal{A}$} = $\sigma (\mathcal{A})$ = $\cap \{\mathcal{G} : \mathcal{G}$ è
          tribù in $\Omega$ e $\mathcal{A} \subseteq \mathcal{G}\}$
    \end{center}
    quindi il prodotto $\Omega\SB{1}\times\Omega\SB{2}$, la tribù sarà:
    \begin{center}
        $\mathcal{F}  =  \mathcal{F}\SB{E}\bigotimes\mathcal{F}\SB{E}$ = 
        $\sigma$(\{$E\SB{1}\times E\SB{2}$ : $E\SB{1}\in\mathcal{F}\SB{E}$, $E\SB{2}\in\mathcal{F}\SB{E}$\})


        ($\Omega\SB{1}, \mathcal{F}\SB{1}, P\SB{1}$), ($\Omega\SB{2}, \mathcal{F}\SB{2}, P\SB{2}$)
        \\$\Downarrow$\\
        $\Omega = \Omega\SB{1}\times\Omega\SB{2}$\\
        $\mathcal{F}  =  \mathcal{F}\SB{E}\bigotimes\mathcal{F}\SB{E}$ = 
        $\sigma$(\{$E\SB{1}\times E\SB{2}$ : $E\SB{1}\in\mathcal{F}\SB{E}$, $E\SB{2}\in\mathcal{F}\SB{E}$\})\\
        P : P(E\SB{1}$\times$E\SB{2}) = P\SB{1}(E\SB{1}) $\cdot$ P\SB{2}(E\SB{2})
    \end{center}

    Quindi:\\
    con un numero finito di esperimenti \{($\Omega\SB{i}, \mathcal{F}\SB{i}, P\SB{i}$)\}\SB{i$\in$I}
    allora lo spazio prodotto ha forma:\\
    $\Omega  =  \bigotimes_{i\in I} \Omega_{i}$\\
    $\mathcal{F}  =  \bigotimes_{i\in I} \mathcal{F}_{i}  =  \sigma (\Pi_{i\in I}\ E_{i}: E_{i}\in\mathcal{F}_{i}$ e $\exists$n tc $\forall j\geq n$ E\SB{j}=$\Omega_{i}$)\\
    P = $\bigotimes_{i\in I}$ P\SB{i} cioé P($\Pi_{i\in I}$ E\SB{i}) = $\Pi_{i\in I}$P\SB{i}(E\SB{i})

\end{enumerate}
\end{document}