\documentclass[12pt, a4paper]{article}
\usepackage[utf8]{inputenc}

\usepackage{graphicx}

\usepackage{geometry}

\usepackage{multicol}

\usepackage{amsmath}
\usepackage{amsfonts}
\usepackage{amssymb}

\geometry{margin=0.6in}


\setlength{\parindent}{0em}
\setlength{\parskip}{1em}


\begin{document}
\section{Database management systems}

Databases are developed in a way to grant data independence, integrity and security, efficient and concurrent
access and reduced application development time.

\subsection{Data models}
A data model are the concepts used to describe data. A schema is the description of a particular 
collection of data.\\The most used model today is the relational model. It bases its self on the concept 
of relations, and every relation has a schema

\subsubsection{Abstraction}
Data is defined in different schemas. The physical schema represents how the data is stored on the disk,
the conceptual schema defines the logical structure, and a view is how users see the data.

Schemas are defined using DDL (Data Definition Language) and the data is modified using DML (Data 
Manipulation Language)

\subsection{ACID}
ACID is the list of properties that a database must ensure. To explaine them we use the example of a 
bank transaction

\subsubsection{Atomicity}
Atomicity is a property caracterized by the all-or-nothing policy. It is applied through a log who keeps

\subsubsection{Consistency}
The databse must be consistent before and after the transaction

\subsubsection{Isolation}
multiple transactions occur independently without interference

\subsubsection{Durability}
THe changes of a successful transaction occurs even if the system failure occurs

\newpage
\section{E/R Model}

\subsection{Entity}
an entity is a real world object which is described using a set of attributes. A collection of similar
entities is a entity set

\subsubsection*{weak entities}
This are entities that depend to another entity. The weak entity doesn't make sense to exist by itself. 
This entities are rappresented through a thick line around the entity and they use a thick arrow towards
the entity they depend

\subsection{Relations}
It is a connetion between two or more entity sets. \\The relation can use different multiplicity, which
are many to many, many to one and one to one. We use the wisconsin representation system, which is 
structured like this:\\ we use an arrow to point to "one", a think arrow to represent "exactly one", a 
line to represent "mmany to many" and a thick line for "at least one"

\subsubsection{Aggregation}
is used when we have to model a relationship involving entity sets and a relationship set

\subsection{Subclasses}
it's a special case of subentity, it's represented with a triangle

\subsubsection*{E/R Subclasses}
Subclasses form a tree of one-one inheritance that use isa relationships. 

\subsection{Keys}
This are unique identifiers of an entity and are represented through an underlined attribute. They can
be super keys (keys in general), candidate key (minimal super key) and primary key (the chosen key)

\subsection{Schemas}

\newpage
\section{Relational model}
it's the most used model thanks to the characteristic to be easier and faster than other models.

\subsubsection*{Definitions}
a relational database is a set of relations, which are made of instances, or tables, and schemas, which specifies name 
of the relation, and name plus type of each column. In the tables, rows equals to the cardinality and are called tuples,
the fields are called degrees

\subsection{Integrity contraints}
A condition must be true for any instance of the database.

Keys can be contraints because no distinct tuples can have the same values 




\end{document}

