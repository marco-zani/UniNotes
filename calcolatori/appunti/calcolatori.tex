\documentclass[12pt, a4paper]{article}
\usepackage[utf8]{inputenc}
\usepackage{amsmath}

\def\separator{\begin{center}    \rule{100pt}{0.5pt}\end{center}}

\setlength{\parindent}{0em}
\setlength{\parskip}{1em}



\begin{document}

\section{INTRODUZIONE}
\subsection{Astrazioni}
Per isolare i vari livelli di un computer (hardware, kernel, SO), ogni strato fornisce delle interfacce (o set di istruzioni) con la quale interagire al livello sottostante.
Le più importanti sono:
\begin{itemize}
    \item ISA (Instruction Set Architecture) $\rightarrow$ insieme istruzioni macchina
    \item ABI (Application Binary Interface) $\rightarrow$ interfaccia delle applicazioni
\end{itemize}

\subsection{prestazioni}
Le prestazioni si misurano con il \textbf{tempo di esecuzione di un programma} che è composto da 3 variabili:
\textbf{numero di istruzioni}, \textbf{cicli di clock per istruzione} e \textbf{frequenza di clock}
\begin{center}
    tempo di CPU = $\frac{numero\ istruzioni\ \times \ CPI}{frequenza\ di\ clock}$
\end{center}

\subsection{ISA}
 - influisce direttamente sul tempo di cpu (un isa è progettata per frequenze di clock più spinte, è esplicitato il numero di cicli per un'instruzione, e il numero di istruzioni stesso per compiere un'operazione)
 le ISA che affronteremo:
 \begin{itemize}
    \item RISC-V (RISC): cloud computing e sistemi embedded
    \item Intel (CISC): PC
    \item ARM (A-RISC): embedded e mobile
 \end{itemize}


\section{ARITMETICA DEI CALCOLATORI}

\subsection{Basi}
 L'aritmetica nei calcolatori viene fatta su base binaria, quindi bisogna eseguire delle conversioni da una base all'altra.
 per un passaggio da decimale a binario, prendo il numero in decimale e calcolo ricorsivamente il modulo di 2, da qui tengo la parte il resto.
 Quando arrivo a 1, il mio numero in binario saranno i resti letti al contrario.
 
 L'addizione in decimale e binario rimane uguale, mentre le moltiplicazioni seguono questo algoritmo: per ogni 0 del moltiplicatore
 mi sposto di un posto a sinistra, mentre con un 1 copio il moltiplicando. Alla fine addiziono tutto.

\subsection{Codifica}
\subsubsection{Numeri negativi}
Esistono 3 metodi per rappresentare i numeri negativi: \textbf{modulo e segno}, \textbf{complemento a 1}, \textbf{complemento a 2}. 
Con tutti questi metodi, il bit più a sinistra rappresenta il segno

\textbf{modulo e segno:} il più semplice, con il bit più a sinistra rappresenta il segno (1 per -, 0 per +)

\textbf{complemento a 1:} un numero positivo viene rappresento come valore assoluto, mentre uno negativo
 lo rappresento con complimento a 1. \\
    I metodi del complemento sono 2:\\
    - cambio tutti i bit da 1 a 0 e da 0 a 1\\
    - sottraggo il numero a un numero della stessa lunghezza con tutti i bit a 1 
    (11111 - 00101 = 11010)
   
\textbf{Somma e sottrazione con complemento a 1:}
    i numeri negativi li rappresento con complemento a 1 e poi sommo i 2 numeri.Il riporto sul bit più significativo
    lo sommo al risultato. 
    \begin{center}
         6-3 $\rightarrow$ 6 + (-3) $\rightarrow$ 00110 + 11100 $\rightarrow$ 00010 con riporto di 1 $\rightarrow$
         00010 + 1 = 00011 = 3
    \end{center}

\textbf{complemento a 2:}
per eseguire un complemento a 2 si possono usare 3 metodi:
\\ - metodo 1: dato un numero trovo il primo bit a 1 partendo da destra, poi eseguo il complemento a 1 per tutti i bit successivi
\\ - metodo 2: eseguo il complemento a uno del numero, e poi addiziono 1
\\ - metodo 3: prendo il numero con primo bit di sinistra a 1 e resto a zero e di lunghezza pari al numero che ci interessa, infine sottraggo quest'ultimo

Il complemento a 2 risulta più conveniente per le somme, la codifica dello zero è unica e nell'operazione inversa, infatto per eseguire quest'ultima
si osserva il primo bit. Se è pari a 0, allora converto normalmente, se no eseguo il complemento a 2 e sommo 1, poi converto in base 10
\\Il vantaggio con il complemento a 2 si guadagna un valore negativo. Sfortunatamento però si posson oanche generare degli overflow.
Bisogna sempre controllare se da somma di positiva risulta in negativo o se da somma di negativi risulta un positivo
 


\subsubsection{Rappresentazione numeri reali}
\textbf{virgola fissa}
  - si dedica una parte della stringa di bit come parte intera e parte frazionaria
  - genericamente trattiamo il numero come intero e poi moltiplichiamo per -n (dove n rappresenta le cifre decimali)
  - non ci sono errori di approsimazione, ma risulta difficile gestire numeri particolarmente grandi o piccoli
  - per la conversione della parte decimale moltiplico per 2 e considero le parti intere

\textbf{virgola mobile}
  - un numero reale viene suddiviso come mantissa ed esponente

\end{document}