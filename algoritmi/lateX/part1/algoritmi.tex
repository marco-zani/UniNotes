\documentclass[12pt, a4paper]{article}
\usepackage[utf8]{inputenc}

\usepackage{graphicx}

\usepackage{geometry}

\usepackage{multicol}
\usepackage{listings}


\usepackage{amsmath}
\usepackage{amsfonts}
\usepackage{amssymb}

\geometry{margin=0.6in}


\setlength{\parindent}{0em}
\setlength{\parskip}{1em}

\title{Algoritmi e strutture Dati - Parte 1}

\begin{document}    

\section{Introduzione}
Gli algoritmi sono i procedimenti che ci permettono di risolvere dei problemi computazionali, gli scriviamo in 
pseudocodice per una descrizione formale e viene valutato tramite la sua correttezza e efficenza. 

NB: Alcuni problemi non possono essere risolti in modo corretto, quindi vengono approssimati. Lo stesso vale per
l'efficenza, che si sposta su una soluzione ottimale

L'efficenza di un algoritmo viene osservata tramite l'analisi delle risorse impiegate in funzione della dimensione
e tipologia dell'input. Generalmente parliamo di spazio di memoria, quantità di banda e tempo. In questo caso
però il tempo viene indicato come il numero di operazioni rilevanti compiute dall'algoritmo, per esempio i
confronti

La correttezza di un algoritmo viene valutata con le invarianti (normali, di ciclo e di classe), ovvero condizioni
sempre vere in punti specifici del programma. Il caso della invariante di ciclo è particolare perché va valutata
all'inizializzazione del ciclo, durante le iterazioni (conservazione) e alla conclusione

Altre proprietà secondarie sono semplicità, robustezza, modularità, espandibilità, manutenibilità, ...

\newpage
\section{Analisi di algoritmi}
\subsubsection*{Notazione}
Noi rappresentiamo con $O(g(n))$ l'insieme delle funzioni tali che $g(n)$ (funzione costo) sia il limite asintotico
superiore, e lo chiameremo "O grande di $g(n)$". $\Omega$ invece rappresenta il limite asintotico inferiore.
La nostra funzione può essere anche rappresentata con $\Theta(g(n))$.

\subsection{Complessità di un algoritmo}

\subsection{Proprietà della notazione asintotica}

\subsubsection*{Dualità}
\begin{center}
    $f(n) = O(g(n)) \Leftrightarrow g(n) = \Omega(f(n))$
\end{center}

\subsubsection*{Eliminazione delle costanti}
\begin{center}
    $f(n) = O(g(n)) \Leftrightarrow af(n) = O(g(n)),\forall a > 0 $\\
    $f(n) = \Omega(g(n)) \Leftrightarrow af(n) = \Omega(g(n)),\forall a > 0$
\end{center}

\subsubsection*{Sommatoria}
\begin{center}
    $1(n) = O(g1(n)),f2(n) = O(g2(n)) \Rightarrow f1(n) + f2(n) = O(max(g1(n),g2(n)))
    f1(n) = \Omega (g1(n)),f2(n) = \Omega (g2(n)) \Rightarrow f1(n) + f2(n) = \Omega (max(g1(n),g2(n)))$
\end{center}

\subsubsection*{Prodotto}
\begin{center}
    $f1(n) = O(g1(n)),f2(n) = O(g2(n)) \Rightarrow f1(n) \cdot f2(n) = O(g1(n) \cdot g2(n))$\\
    $f1(n) = \Omega (g1(n)),f2(n) = \Omega (g2(n)) \Rightarrow f1(n) \cdot f2(n) = \Omega (g1(n) \cdot g2(n))$
\end{center}

\subsubsection*{Simmetria}
\begin{center}
    $f(n) = \Theta(g(n)) \Leftrightarrow g(n) = \Theta(f(n))$
\end{center}

\subsubsection*{Transitività}
\begin{center}
    $f(n) = O(g(n)),g(n) = O(h(n)) \Rightarrow f(n) = O(h(n))$
\end{center}

\end{document}