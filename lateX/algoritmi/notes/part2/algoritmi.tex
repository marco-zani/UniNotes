\documentclass[12pt, a4paper]{article}
\usepackage[utf8]{inputenc}

\usepackage{graphicx}

\usepackage{geometry}

\usepackage{multicol}
\usepackage{listings}


\usepackage{amsmath}
\usepackage{amsfonts}
\usepackage{amssymb}

\geometry{margin=0.6in}


\setlength{\parindent}{0em}
\setlength{\parskip}{1em}

\title{Algoritmi e strutture Dati - Parte 2}

\begin{document}    

\section{Programmazione dinamica}
Per affrontare un problema ci sono molteplici soluzioni: divide-et-impera, tecnica greedy, ricerca locale, 
backtrack, algoritmi probabilistici e la programmazione dinamica

La programmazione dinamica, come in divide-et-impera, spezza ricorsivamente il problema, ma risolve i problema una
volta sola, memorizzando il risultato con accesso diretto

%%%%%%% GRAFICO %%%%%%%

Quindi, il risultato di un particolare problema viene salvato in una \textbf{tabella DP} che può essere un vettore,
una matrice o un dizionario. Nel dettaglio vengono salvate i casi base direttamente nella tabella, e poi 
vengono risolti con iterazione bottom-up

\end{document}