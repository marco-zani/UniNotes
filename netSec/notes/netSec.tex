\documentclass[12pt, a4paper]{article}
\usepackage[utf8]{inputenc}

\usepackage{graphicx}

\usepackage{geometry}

\usepackage{multicol}
\usepackage{listings}


\usepackage{amsmath}
\usepackage{amsfonts}
\usepackage{amssymb}

\geometry{margin=0.6in}


\setlength{\parindent}{0em}
\setlength{\parskip}{1em}

\title{NetSec}

\begin{document}
\section{Introduction}
the terms information, computer and network security get confused, but the difference is important
\begin{itemize}
    \item information security regards all the the prevention of any damage involving data, in any format
    \item Computer security regards the protection of the phisical machine and the data it contains
    \item Network security regards the protection of the network and infrastructure of a company
\end{itemize}
cybersecurity is the ability to protect the use of cyberspace (internet, private networks,...) from cyber
attacks

\subsection{Remarks on security}
\subsubsection*{Adversaries}
Security is characterized by protection against malicious adversaries, with different motivations (money
or glory), different capabilities and objectives. This details compone the modelling of a malicious 
adversary

\subsubsection*{Mitigation}
To mitigate a threat, various strategies are aviable, reguarding people, processes and technologies.
This controls are classified in preventive, detective and corrective.\\What conditionates which tactic 
to select is risk management, or the process to identify the probability of an attack and the extention 
of the subsequent damage

\subsubsection*{Trust}
A library, or dependency, or every third party product to a software, requires trust and dependability,
also known as the ability to avoid failures.

\subsubsection*{Residual Risks}
All security controls can mitigate an attack, but do not avoid it completely. In some cases they can contain more vulnerabilities

\subsection{Cia Triad}
The most important security properties are Confifentiality, Integrity and Availability

\subsubsection*{Confidentiality}
The main objective of confidentiality is to protect personal privacy and proprietary information from unauthorized indiviaduals.
Confidentiality covers data in storage, during processing and while in transit

\subsubsection*{Integrity}
In this case our objective is to avoid unauthorized modification or destruction of data, including the authenticity and non-repudiation

\subsubsection*{Availability}
Ensuring timely and reliable access to data and services by authorized users

\subsection{Risk}
\subsubsection*{Vulnerability}
A vulnerability is a weakness or flaw in a system, procedure, design or network that can be exploited by a threat source.
They are characterized by the difficulty in identifying and exploiting them

\subsubsection*{Threat}
A threat is any circumstance or event that could impact organiazational operations or assets via unauthorized access, destruction and 
modification of data or denial of service. They are characterized by the propensity to attack (intent) and the ability to successfully do it (capability)

\subsubsection*{Risk}
the probability that a threat will exploit a system vulnerability is called risk, and it's calculated in function of the impact and 
the likelihood of occurrence.

The risk is anilized though a matrix $5x5$ which uses the the severity of the impact and the likelihood to happen as axis and gives point based on the
multiplication of the axes

\subsection{Secutity policy, service and mechanism}
This three things are, in order, the rules set by an organization to protect data, the capabilities to support one or more security requirements and the
devices and functions used to protect data and provide security services

\newpage
\section{Security in perspective}
\subsection{Attacks and impact}
Every component of the CIA triad is an objective of attacks. The violation of confidentiality leads to a disclosure of your personal
information. In the integrity case impacts on the trustworthiness of data and the violation of availability brings economic damage

Further difficulties come from the ever extending attack surface and the acquisition of cyber-intelligence information. The second case 
is an issue because corporations don't usually share informations about their security system or the vulnerabilities they found.

\subsection{Security and privacy}
They are two different but overlapping properties. Privacy can be also used to indicate confidentiality or controlled sharing.

\subsubsection*{Non-repudiation}
It's a protection against an individual falsely denying to having performed a particular action. The opposite is plausible 
deniability. This two cases are a conflict of privacy and security because one violates the other.

\subsection{Legal constraints}
The GDPR (General Data Protection Regulation) is a regulation that applies all european companies and non european companies
that process europeans personal data. It protects all the data reguarding genetic, metal, cultural, economic and social identities.
When it's not followed it applies penalities like fines up to 20 million euros or the 4\% of annual global revenue of a company.
The GDPR requires that products, systems and processes follow privacy-by-design concepts and a risk-based approach to cybersecurity 

\subsection{Security and human factors}
The human factor can be a great vulnerability in a system, so it's important to provide security training for the employees.
The usual concept to follow when creating security policies for the human factor is:
\begin{center}
    Make it easy to do the right thing, hard to do the wrong thing and easy to recover when the wrong thing happens
\end{center}

\newpage
\section{Authentication - Passwords and more}
Passwords are the main human computer authentication system, but more secure and userfriendly alternatives are researched.
In a secure system  the user identity is used as a parameter for access controll and all relevant events are logged.

The current situation has found that there isn't a perfect authentication system, but a combination of password, OTP and other
techniques is more likely to succeed

\subsection{Lessons learned}
Passwords were created for protection against jokes and abuse of resources, but it was easy to guess passwords or to find the 
passwords stored in clear. To prevent this password hashing and salting were invented. After a major attack (morris worm) password
were stored in a shadowfile readable only by the root user

\subsection{Passwords and the web}
With the arrival of the internet, new issues with passwords have been discovered: the reuse of the same password for different 
services, phishing and the need to reset forgotten passwords. They have been made some attempts to outsource the authentication
to third party services, and smartphones increase security using the two factors authentication.

\subsection{User authentication and digital identity}
When logging on to a computer you enter username (announcing who you are) and password (proving to be who you claim to be). This
type of authentication is called \textbf{user authentication}, in this case, \textbf{by password}.

BOOTSTRAPING

The process of identity proofing follows 3 steps:
\begin{itemize}
    \item Resolution: collection of data/evidence called Personal Identifiable Information (PII)
    \item Validation: processing of the evidence by an authoritative source, searching for coherence of the information and 
    complaiance against standards
    \item Verification: the evidence is verified and the user is allowed to proceed
\end{itemize}
There are different levels of certainty of one user identification, these are called Identity Assurance Levels (IALs), and 
are divided in 3 levels. With IAL 1, all attributes are self-asserted by the user or should be treated as self-asserted.
at level 2 requires either remothe or in-person identification. Lastly, at third level, is required the verification by
an authorized representative.

\subsection{Attacks and mitigations}
One way to get a password is to guess it. This opens the way to two attack types: the brute force, which tries all possible 
combinations of symbols, and the smarter dictionary, which uses a list of commonly used passwords.\\This type of attacks
can't be prevented, but the probability to be successful can be reduced. The mitigations are:
\begin{itemize}
    \item change default passwords
    \item forbid commonly used passwords
    \item limit the number of attempts
    \item don't use password hints or knowledge based
    \item incourage the use of password generators and managers
\end{itemize}

\subsubsection*{Spoofing}
Password spoofing is the use of programs with fake login interfaces that captures user and password when a legittimate user 
gets bamboozled. The countermeadure is to implement mutual authentication between user and OS

\subsubsection*{Phishing}
an attacker impersonates the system to trick a unser into releasing sensible data. It can also use social engineering 

\subsubsection*{Credential stuffing}
Is the use of information from compromised databases to be applied on other services

\subsubsection*{Password file protection}
To maintain confidentiality and integrity the file could use cryptographic protection, access controll enforced by the OS,
and further measures to slow dictionary attacks

\subsection{Hashing and salting}
The Hash is a one way function easy to compute but hard to reverse. Password usually get stored in the form of hash.
The characteristics of an hash function are the easy computation, compression, one-way and collision resistance.

NB: strong collision resistance is the property to be improbable to produce 2 messages with the same hash code. This can 
be problematic if your storing passwords. Weak collision is when two passwords retrieve the same hash in the login. This 
could lead to an exploit of the system security

The principal algoritms in use are RIPEMD-160 and SHA-256 (others have been cracked like MD4, MD5, SHA-1). 
Hashing can be broken through the use of dictionary or rainbow (hash dictionaries) attacks.

To increase the security salting has been introduced to the hashes. Salting is the practice of appending random values
to the end of the password before hashing

\subsection{Extensions for password based authentication}
Multi factor authentication is characterized by the addition of some properties like knowledge, possession or inherence.
The most common is two factor authentication (2FA) 

\subsubsection{Time based one time passwords}
it uses an algoritm to generate the same number on the server and the host, then the user sends the value that is shown 
by the host, and if the two number are equals, the user is authorized. This methods still hase some weaknesses, like the
value could be phished and replayed, or the algoritm could be stolen and used by the attacker to create new OTP

Smartphones are widely used in MFA, they can provide additional mobile systems to generate password codes, but they also 
represent vulnerabilities thanks to weak protection of text messages, SIM cloning and the event that the smartphone gets 
stolen

\subsection{Assurance level (NIST)}
\begin{enumerate}
    \item requires at least single factor authentication, provides some assurance the claimant controls the authenticator
    \item requires two factor authentication, provides high confidence that the claimant controls the authenticator
    \item requires proof of possession, providing very high confidence that the claimant controls the authenticator
\end{enumerate}

\subsection{Outsourcing authentication}
Securing efficiently the authentication on a system cn be difficult for smaller organizations, so he solution is to 
delegate authentication to trusted third party providers. 

Single Sign On (SSO) is one method the uses outsourcing. It's a mechanism that collects your password on the first
login, and automatically repeats it on the successives logins. This increases the usabilty of a system and allows the 
use of more complex password, at the cost of just one password to compromise.\\This mechanism is also an example of the balance
between convenience and security

\newpage
\section{Cryptography}
\subsection{Definitions}
\begin{itemize}
    \item Cryptography is the science and study of secret writing
    \item Cryptanalysis is the science and study of methods of breaking ciphers
    \item Cryptology is the combination of cryptography and cryptanalysis
    \item Cryptosystem is the combination of:
    \subitem $E$ is an encryption algoritm
    \subitem $D$ is a decryption algoritm
    \subitem $M$ is a set of plain texts
    \subitem $K$ is the set of keys
    \subitem $C$ is the set of ciphertexts
    \subsubitem we can define $E$ and $D$ as $E:M\times K\to C$ and $D:C\times K\to M$
    \item Kerckhoff's principle: the algoritm should not be the secret, but the key
\end{itemize}

\subsubsection{Keys}
Are one input to a cryptographic algorithm and the most important part to keep secret. the set of all the possible keys
is called keyspace, and the variance of a key is measured with entropy. Key are important because they're computationally
secure 

During a connection is common to change the key to limit the amount of data that could be compromised with a single key.
So having a strong distribution system is essential 

Cryptography is just one part to make a system secure, the keys must stored securely

\subsection{Encryption}
The encryption process can be applied on blocks of same lenght or stream of data. The transformation could be through:
\begin{itemize}
    \item Substitution: each element is mapped into another element
    \item Transposition: elements are rearranged
\end{itemize}

\subsubsection{Caesar cipher (ROTk)}
It's an algoritm based on rotation, matematicaly:
\begin{center}
    $E(x)=(x+k)(mod\ 26)\qquad x\in P$\\
    $D(x)=(x-k)(mod\ 26)\qquad x\in C$
\end{center}
it's one of the easiest cipher to break

\subsubsection{Vigenere cipher}
it's an evolution of the Caesare cipher, it uses the same principle but instead of rotating with the same value, it's
used a repeated key

\subsubsection{Columnar cipher}
The message is written on rows of a table with fixed column lenght defined by the key (the empty spaces are filled with
nulls). In the end the columns are scrambled based on the key and the message gets rewritten from each column

\subsection{Modern encryption}
With the invention of computers, the algoritms had to evolve because they were too easy to brute force. In this way, new 
methods arise 

\subsection{Symmetric keys}
A single key is used to encrypt and decrypt and is share between all intended receivers. it is categorized in two types: 
stream cyphers and block cyphers. One detail on block cyphers they use both substitution an transposition through S-boxes
to obtain confusion and P-boxes to obtain diffusion. This two properties are responsable of the dependency of the key and 
the fact the if we change one bit, the entire message cyphertext changes

\subsubsection*{Stream ciphers}
converts plaintext one bit at a time with xor and the key. it's more efficient and it's used for realtime applications

\subsubsection*{Block ciphers}
breaks the message in successive blocks and enciphers them with the same key

\subsubsection{Feistel}
also called DES, it's a block cipher and follows these procedure
\begin{enumerate}
    \item splits the blocks in two
    \item applies substitution to one half, and the output is xored with the other half
    \item it swaps the two halves
\end{enumerate}

\subsubsection{AES}
it's the successor to DES. The encryption process is based on a series of table lookups and xor operations. The difference 
with DES is that the encryption and decryption algorithms are separately implemented, even if closely related. It also
mounts larger keys

\subsection{Asymmetric keys}






\newpage
\section{Wrap-up}
\subsection{Security in perspective}
\begin{itemize}
    \item Attacks violate one or more of the CIA properties
    \item Identifying which properties are violated help in
    \subitem understanding the impact and
    \subitem identifying which security services should be used to deploy security mechanisms for risk mitigation
    \item Ensuring security is a difficult task because of several issues
    \subitem attribution
    \subitem scale of and borderless attacks
    \subitem growing attack surface
    \subitem lack of collaboration among victims
    \item Security and privacy overlaps but are different and sometimes conflicting
    \item Privacy preserving techniques are relevant also from a legal point of view
    \subitem The GDPR encapsulates a risk based approach to privacy
    \item Security is not purely technical, human factors play a crucial role
    \subitem Hence the importance of security awareness and security training to improve the security posture
\end{itemize}
\subsection{Passwords and authentication}
\begin{itemize}
    \item Authentication amounts to verifying the identity of a user, process, or device, often
    as a prerequisite to allowing access to resources in an information system
    \item Here we focused on user authentication
    \item Passwords are one of the most widespread and accepted method for user
    authentication despite their shortcomings
    \item Adequate security mechanisms should be put in place to protect stored passwords
    \subitem Hashing & salting, access control, ...
    \item Authentication is only one phase of the identity management lifecycle
    \subitem Also other phases (e.g., identity proofing) may have weaknesses
    \item Given the importance and difficulty of identity management, it is frequently
    outsourced to trusted third parties with advantages and disadvantages also from
    the viewpoint of security (and privacy)
\end{itemize}


\end{document}
