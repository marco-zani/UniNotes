\documentclass[12pt, a4paper]{article}
\usepackage[utf8]{inputenc}

\usepackage{graphicx}

\usepackage{geometry}

\usepackage{multicol}
\usepackage{listings}


\usepackage{amsmath}
\usepackage{amsfonts}
\usepackage{amssymb}

\geometry{margin=0.6in}


\setlength{\parindent}{0em}
\setlength{\parskip}{1em}

\title{NetSec}

\begin{document}
\section{Introduction}
the terms information, computer and network security get confused, but the difference is important
\begin{itemize}
    \item information security regards all the the prevention of any damage involving data, in any format
    \item Computer security regards the protection of the phisical machine and the data it contains
    \item Network security regards the protection of the network and infrastructure of a company
\end{itemize}
cybersecurity is the ability to protect the use of cyberspace (internet, private networks,...) from cyber
attacks

\subsection{Remarks on security}
\subsubsection*{Adversaries}
Security is characterized by protection against malicious adversaries, with different motivations (money
or glory), different capabilities and objectives. This details compone the modelling of a malicious 
adversary

\subsubsection*{Mitigation}
To mitigate a threat, various strategies are aviable, reguarding people, processes and technologies.
This controls are classified in preventive, detective and corrective.\\What conditionates which tactic 
to select is risk management, or the process to identify the probability of an attack and the extention 
of the subsequent damage

\subsubsection*{Trust}
A library, or dependency, or every third party product to a software, requires trust and dependability,
also known as the ability to avoid failures.

\subsubsection*{Residual Risks}
All security controls can mitigate an attack, but do not avoid it completely. In some cases they can contain more vulnerabilities

\subsection{Cia Triad}
The most important security properties are Confifentiality, Integrity and Availability

\subsubsection*{Confidentiality}
The main objective of confidentiality is to protect personal privacy and proprietary information from unauthorized indiviaduals.
Confidentiality covers data in storage, during processing and while in transit

\subsubsection*{Integrity}
In this case our objective is to avoid unauthorized modification or destruction of data, including the authenticity and non-repudiation

\subsubsection*{Availability}
Ensuring timely and reliable access to data and services by authorized users

\subsection{Risk}
\subsubsection*{Vulnerability}
A vulnerability is a weakness or flaw in a system, procedure, design or network that can be exploited by a threat source.
They are characterized by the difficulty in identifying and exploiting them

\subsubsection*{Threat}
A threat is any circumstance or event that could impact organiazational operations or assets via unauthorized access, destruction and 
modification of data or denial of service. They are characterized by the propensity to attack (intent) and the ability to successfully do it (capability)

\subsubsection*{Risk}
the probability that a threat will exploit a system vulnerability is called risk, and it's calculated in function of the impact and 
the likelihood of occurrence.

The risk is anilized though a matrix $5x5$ which uses the the severity of the impact and the likelihood to happen as axis and gives point based on the
multiplication of the axes

\subsection{Secutity policy, service and mechanism}
This three things are, in order, the rules set by an organization to protect data, the capabilities to support one or more security requirements and the
devices and functions used to protect data and provide security services

\newpage
\section{Security in perspective}
\subsection{Attacks and impact}
Every component of the CIA triad is an objective of attacks. The violation of confidentiality leads to a disclosure of your personal
information. In the integrity case impacts on the trustworthiness of data and the violation of availability brings economic damage

Further difficulties come from the ever extending attack surface and the acquisition of cyber-intelligence information. The second case 
is an issue because corporations don't usually share informations about their security system or the vulnerabilities they found.

\subsection{Security and privacy}
They are two different but overlapping properties. Privacy can be also used to indicate confidentiality or controlled sharing.

\subsubsection*{Non-repudiation}
It's a protection against an individual falsely denying to having performed a particular action. The opposite is plausible 
deniability. This two cases are a conflict of privacy and security because one violates the other.

\subsection{Legal constraints}
The GDPR (General Data Protection Regulation) is a regulation that applies all european companies and non european companies
that process europeans personal data. It protects all the data reguarding genetic, metal, cultural, economic and social identities.
When it's not followed it applies penalities like fines up to 20 million euros or the 4\% of annual global revenue of a company.
The GDPR requires that products, systems and processes follow privacy-by-design concepts and a risk-based approach to cybersecurity 

\subsection{Security and human factors}
The human factor can be a great vulnerability in a system, so it's important to provide security training for the employees.
The usual concept to follow when creating security policies for the human factor is:
\begin{center}
    Make it easy to do the right thing, hard to do the wrong thing and easy to recover when the wrong thing happens
\end{center}

\newpage
\section{Authentication - Passwords and more}
Passwords are the main human computer authentication system, but more secure and userfriendly alternatives are researched.
In a secure system  the user identity is used as a parameter for access controll and all relevant events are logged.

The current situation has found that there isn't a perfect authentication system, but a combination of password, OTP and other
techniques is more likely to succeed

\subsection{Lessons learned}
Passwords were created for protection against jokes and abuse of resources, but it was easy to guess passwords or to find the 
passwords stored in clear. To prevent this password hashing and salting were invented. After a major attack (morris worm) password
were stored in a shadowfile readable only by the root user




\newpage
\section{Wrap-up}
\subsection{Security in perspective}
\begin{itemize}
    \item Attacks violate one or more of the CIA properties
    \item Identifying which properties are violated help in
    \subitem understanding the impact and
    \subitem identifying which security services should be used to deploy security mechanisms for risk mitigation
    \item Ensuring security is a difficult task because of several issues
    \subitem attribution
    \subitem scale of and borderless attacks
    \subitem growing attack surface
    \subitem lack of collaboration among victims
    \item Security and privacy overlaps but are different and sometimes conflicting
    \item Privacy preserving techniques are relevant also from a legal point of view
    \subitem The GDPR encapsulates a risk based approach to privacy
    \item Security is not purely technical, human factors play a crucial role
    \subitem Hence the importance of security awareness and security training to improve the security posture
\end{itemize}


\end{document}
